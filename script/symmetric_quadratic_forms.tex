\section{Symmetrische Bilinearformen und quadratische Formen}










\subsection{Definition quadratischer Formen}


\begin{definition}
  Ist $V$ ein $K$-Vektorraum und $\bracket{\cdot, \cdot} \colon V \times V \to K$ eine symmetrische Bilinearform, so ist
  \[
    q \colon V \to K,
    \quad
    v \mapsto \bracket{v, v}
  \]
  die zu $\bracket{\cdot, \cdot}$ gehörige \emph{quadratische Form}.
  
  Eine Abbildung $q' \colon V \to K$ heißt \emph{quadratische Form}, wenn es eine symmetrische Bilinearform $\bracket{\cdot, \cdot}' \colon V \times V \to K$ gibt, so dass $q'$ die zu $\bracket{\cdot, \cdot}'$ gehörige quadratische Form ist.
\end{definition}


\begin{lemma}
  Ist $V$ ein $K$-Vektorraum mit $\ringchar K \neq 2$, $\bracket{\cdot, \cdot} \colon V \times V \to K$ eine symmetrische Bilinearform und $q \colon V \to K$ die zugehörige quadratische Form, so ist
  \[
    \bracket{v, w} = \frac{q(v+w)-q(v)-q(w)}{2}
    \quad
    \text{für alle $v, w \in V$}.
  \]
\end{lemma}


\begin{corollary}
  Ist $V$ ein $K$-Vektorraum mit $\ringchar(K) \neq 2$, so ist die Abbildung
  \begin{align*}
    \{ \text{Bilinearformen $V \times V \to K$} \}
    &\to
    \{ \text{quadratische Formen $V \to K$} \},
    \\
    \bracket{\cdot, \cdot}
    &\mapsto
    (v \mapsto \bracket{v,v})
  \end{align*}
  eine Bijektion.
\end{corollary}


\begin{definition}
  Es sei $\beta \colon V \times V \to K$ eine symmetrische Bilinearform.
  \begin{enumerate}[leftmargin=*, label=\roman*)]
    \item
      Zwei Vektoren $u, w \in V$ sind \emph{orthogonal (zueinander)} bezüglich $\beta$ falls $\beta(u,w) = 0$.
    \item
      Zwei Untervektorräume $U, W \subseteq V$ heißen \emph{orthogonal (zueinander)} bezüglich $\beta$ falls $\beta(u,w) = 0$ für alle $u \in U$ und $w \in W$.
    \item
      Ist $U \subseteq V$ ein Untervektorraum, so ist der Untervektorraum
      \[
        U^\perp
        =
        \{
          v \in V
          \mid
          \text{$\beta(u,v) = 0$ für alle $u \in U$}
        \}
      \]
      das \emph{orthogonale Komplement} von $U$ in $V$ bezüglich $\beta$.
    \item
      Der Untervektorraum
      \[
        \rad(\beta)
        \coloneqq
        \{v \in V \mid \text{$\beta(u,v) = 0$ für alle $u \in V$}\}
      \]
      ist das \emph{Radikal} von $\beta$.
  \end{enumerate}
\end{definition}












\subsection{Darstellende Matrizen für Bilinearformen}


\begin{lemma}
  Es sei $V$ ein endlichdimensionaler $K$-Vektorraum, $\basis{C} = (v_1, \dotsc, v_n)$ eine geordnete Basis von $V$ und $\basis{C}^* = (v_1^*, \dotsc, v_n^*)$ die duale Basis von $V^*$.
  Für eine Matrix $B \in \Mat_n(K)$ und Bilinearform $\beta \colon V \times V \to K$ sind die folgende Bedingungen äquivalent:
  \begin{enumerate}[leftmargin=*, label=\roman*)]
    \item
      Für alle $i,j = 1, \dotsc, n$ ist $B_{ij} = \beta(v_i, v_j)$.
    \item
      Für die lineare Abbildung
      \[
        \Phi \colon V \to V^*,
        \quad
        v \mapsto \beta(-,v)
      \]
      ist $B = \Mat_{\basis{C}, \basis{C}^*}(\Phi)$.
  \end{enumerate}
\end{lemma}


\begin{definition}
  Ist $V$ ein endlichdimensionaler $K$-Vektorraum mit geordneter Basis $\basis{C} = (v_1, \dotsc, v_n)$, und $\beta \colon V \times V \to K$ eine Bilinearform, so heißt die Matrix $B \in \Mat_n(K)$ mit
  \[
    B_{ij} = \beta(v_i, v_j)
    \quad
    \text{für alle $i,j = 1, \dotsc, n$}
  \]
  die \emph{darstellende Matrix} von $\beta$ bezüglich der Basis $\basis{C}$, und wird mit $\Mat_\basis{C}(\beta)$ notiert.
\end{definition}


\begin{lemma}
  Ist $V$ ein endlichdimensionaler $K$-Vektorraum und $\basis{C}$ eine geordnete Basis von $V$, so ist die Abbildung
  \[
    \Mat_\basis{C}
    \colon
    \{\text{Bilinearformen $V \times V \to K$}\}
    \to
    \Mat_n(K),
    \quad
    \beta
    \mapsto
    \Mat_\basis{C}(\beta)
  \]
  ein Isomorphismus von $K$-Vektorräumen.
\end{lemma}


\begin{lemma}
  Es sei $\basis{C} = (v_1, \dotsc, v_n)$ eine geordnete Basis eines $K$-Vektorraums $V$ und
  \[
    \Phi \colon V \to K^n,
    \quad
    \sum_{i=1}^n \lambda_i v_i \mapsto \cvector{\lambda_1 \\ \vdots \\ \lambda_n}
  \]
  der induzierte Isomorphismus.
  Ist $\beta \colon V \times V \to K$ eine Bilinearform, so ist
  \[
    \beta(v,w)
    =
    \Phi(v)^T \Mat_\basis{C}(\beta) \Phi(w)
    \quad
    \text{für alle $v, w \in V$}.
  \]
\end{lemma}


\begin{corollary}
  Ist $\beta \colon V \times V \to K$ eine Bilinearform auf einem endlichdimensionalen $K$-Vektorraum $V$, so ist $\beta$ genau dann symmetrisch, wenn für jede geordnete Basis $\basis{C}$ von $V$ die darstellende Matrix $\Mat_\basis{C}(\beta)$ symmetrisch ist.
\end{corollary}


% TODO: Obere Aussagen verbessern.










\subsection{Nicht-entartete Bilinearformen}


\begin{definition}
  Eine symmetrische Bilinearform $\beta \colon V \times V \to K$ heißt \emph{nicht-entartet}, falls es für jedes $v \in V$ mit $v \neq 0$ ein $w \in V$ mit $\beta(w,v) \neq 0$ gibt.
\end{definition}


\begin{proposition}
  Ist $V$ ein $K$-Vektorraum und $\beta \colon V \times V \to K$ eine symmetrische Bilinearform, so sind die folgenden Bedingungen äquivalent:
  \begin{enumerate}[leftmargin=*, label=\roman*)]
    \item
      Die Bilinearform $\beta$ ist nicht-entartet.
    \item
      Es gilt $\rad(\beta) = 0$.
    \item
      Die lineare Abbildung $\Phi \colon V \to V^*$, $v \mapsto \beta(-,v)$ ist injektiv.
  \end{enumerate}
  Ist $V$ endlichdimensional, so kommen die folgenden Bedingungen hinzu:
  \begin{enumerate}[leftmargin=*, label=\roman*), resume]
    \item
      $\Phi$ ist ein Isomorphismus.
    \item
      Für jede geordnete Basis $\basis{C}$ von $V$ ist die darstellende Matrix $\Mat_\basis{C}(\beta)$ invertierbar.
  \end{enumerate}
\end{proposition}


\begin{proposition}
  Es sei $V$ ein endlichdimensionaler $K$-Vektorraum und $\beta \colon V \times V \to K$ eine nicht-entartete symmetrische Bilinearform.
  Ist $U \subseteq V$ ein Untervektorraum, so sind die folgenden Bedingungen äquivalent:
  \begin{enumerate}[leftmargin=*, label=\roman*)]
    \item
      Die Einschränkung $\beta|_{U \times U}$ ist nicht-entartet.
    \item
      Es ist $U \cap U^\perp = 0$.
    \item
      Es ist $V = U \oplus U^\perp$.
  \end{enumerate}
\end{proposition}











\subsection{Normalenformen und Sylvesterscher Trägheitssatz}


\begin{theorem}
  Es sei $V$ ein endlichdimensionaler $K$-Vektorraum mit $\ringchar(K) \neq 2$ und $\beta \colon V \times V \to K$ eine symmetrische Bilinearform.
  Dann gibt es eine geordnete Basis $\basis{C}$ von $V$, so dass
  \[
    \Mat_\basis{C}(\beta)
    =
    \begin{pmatrix}
      \lambda_1 &         &           &   &         &   \\
                & \ddots  &           &   &         &   \\
                &         & \lambda_n &   &         &   \\
                &         &           & 0 &         &   \\
                &         &           &   & \ddots  &   \\
                &         &           &   &         & 0
    \end{pmatrix}
  \]
  mit $\lambda_i \neq 0$ für alle $i = 1, \dotsc, n$.
  Die Anzahl der Nullen auf der Diagonalen ist eindeutig bestimmt und entspricht $\dim \rad(\beta)$.
\end{theorem}


\begin{lemma}
  Ist $\beta \colon V \times V \to K$ eine nicht-entartete Bilinearform mit $V \neq 0$ und $\ringchar(K) \neq 2$, und $q \colon V \to K$ die zugehörige quadratische Form, dann gibt es $v \in V$ mit $q(v) \neq 0$.
\end{lemma}


\begin{corollary}
  Es sei $V$ ein endlichdimensionaler $K$-Vektorraum, wobei $\ringchar(K) \neq 2$ und $K$ quadratisch abgeschlossen ist.
  Ist $\beta \colon V \times V \to K$ eine symmetrische Bilinearform, so gibt es eine geordnete Basis $\basis{C}$ von $V$, so dass
  \[
    \Mat_\basis{C}(\beta)
    =
    \begin{pmatrix}
      1 &         &   &   &         &   \\
        & \ddots  &   &   &         &   \\
        &         & 1 &   &         &   \\
        &         &   & 0 &         &   \\
        &         &   &   & \ddots  &   \\
        &         &   &   &         & 0.
    \end{pmatrix}
  \]
  Die Anzahl der Nullen auf der Diagonalen ist eindeutig bestimmt und entspricht $\dim \rad(\beta)$.
\end{corollary}


\begin{corollary}[Sylvesterscher Trägheitssatz]
  Es sei $V$ ein endlichdimensionaler $\Reals$-Vek\-tor\-raum und $\beta \colon V \times V \to K$ eine symmetrische Bilinearform.
  \begin{enumerate}[leftmargin=*, label=\roman*)]
    \item
      Es eine geordnete Basis $\basis{C}$ von $V$, so dass
      \[
        \Mat_\basis{C}(\beta)
        =
        \begin{pmatrix}
          1 &         &   &     &         &     &   &         &   \\                                                        
            & \ddots  &   &     &         &     &   &         &   \\
            &         & 1 &     &         &     &   &         &   \\
            &         &   & -1  &         &     &   &         &   \\
            &         &   &     & \ddots  &     &   &         &   \\
            &         &   &     &         & -1  &   &         &   \\
            &         &   &     &         &     & 0 &         &   \\
            &         &   &     &         &     &   & \ddots  &   \\
            &         &   &     &         &     &   &         & 0
        \end{pmatrix}
        =
        \begin{pmatrix}
          I_r &       &     \\
              & -I_s  &     \\
              &       & 0_t
        \end{pmatrix}
      \]
      mit $I_n \in \Mat_n(\Reals)$ und $0_m \in \Mat_m(\Reals)$.
    \item
      Ist $\basis{B} = (u_1, \dotsc, u_r, v_1, \dotsc, v_s, w_1, \dotsc, w_t)$ und
      \begin{align*}
        D_+ &\coloneqq \bracket{u_1, \dotsc, u_r},                    \\
        D_- &\coloneqq \bracket{v_1, \dotsc, v_s},                    \\
        S_+ &\coloneqq \bracket{u_1, \dotsc, u_r, w_1, \dotsc, w_t},  \\
        S_- &\coloneqq \bracket{v_1, \dotsc, v_s, w_1, \dotsc, w_t},
      \end{align*}
      so ist $\beta|_{D_+ \times D_+}$ positiv definit, $\beta|_{D_- \times D_-}$ ist negativ definit, $\beta|_{S_+ \times S_+}$ ist positiv semidefinit und $\beta|_{S_- \times S_-}$ ist negativ semidefinit.
      Außerdem ist $\rad(\beta) = \bracket{w_1, \dotsc, w_t}$.
    \item
      Es gilt
      \begin{align*}
        r
        &=
        \max\,
        \{
          \dim W
          \mid
          \text{$W \subseteq V$ ist ein UVR und $\beta|_{W \times W}$ ist positiv definit}.
        \}
        \\
        s
        &=
        \max\,
        \{
          \dim W
          \mid
          \text{$W \subseteq V$ ist ein UVR und $\beta|_{W \times W}$ ist negativ definit}.
        \}
        \\
        r+t
        &=
        \max\,
        \{
          \dim W
          \mid
          \text{$W \subseteq V$ ist ein UVR und $\beta|_{W \times W}$ ist positiv semidefinit}.
        \}
        \\
        s+t
        &=
        \max\,
        \{
          \dim W
          \mid
          \text{$W \subseteq V$ ist ein UVR und $\beta|_{W \times W}$ ist negativ semidefinit}.
        \}
      \end{align*}
      Insbesondere ist das Tupel $(r,s,t)$ durch $\beta$ eindeutig bestimmt.
    \item
      $\beta$ ist genau dann
      \begin{enumerate}[leftmargin=*, label=\alph*)]
        \item
          positiv definit, wenn $r = \dim V$,
        \item
          negativ definit, wenn $s = \dim V$,
        \item
          positiv semidefinit, wenn $r + t = \dim V$,
        \item
          negativ semidefinit, wenn $s + t = \dim V$,
        \item
          nicht-entwartet, wenn $t = 0$.
      \end{enumerate}
  \end{enumerate}
\end{corollary}


% TODO: Definition und Berechnung der Signatur hinzufügen

















































