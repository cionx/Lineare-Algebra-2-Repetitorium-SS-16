\section{Orientierung}


\begin{lemma}\label{lem: possible definition of same orientation}
  Es seien $\basis{B} = (v_1, \dotsc, v_n)$ und $\basis{C} = (w_1, \dotsc, w_n)$ zwei Basen eines $n$-di\-men\-si\-o\-nal\-er $\Reals$-Vektorraums $V$.
  Dann sind die folgenden Bedingungen äquivalent:
  \begin{enumerate}[leftmargin=*, label=\roman*)]
    \item
      Für den eindeutigen Automorphismus $\Phi \colon V \to V$ mit $\Phi(v_i) = w_i$ für alle $i = 1, \dotsc, n$ gilt $\det(\Phi) > 0$.
    \item
      Für alle alternierende $n$-Linearformen $\omega \colon V^k \to \Reals$ gilt
      \[
        \omega(v_1, \dotsc, v_n) > 0
        \iff
        \omega(w_1, \dotsc, w_n) > 0.
      \]
  \end{enumerate}
\end{lemma}


\begin{definition}
  Zwei geordnete Basen eines $\Reals$-Vektorraums $V$ heißen \emph{gleichorientiert}, wenn sie die Bedingungen aus Lemma~\ref{lem: possible definition of same orientation} erfüllen.
\end{definition}


\begin{lemma}
  Ist $V$ ein endlichdimensionaler $\Reals$-Vektorraum, so ist Gleichorientiertheit eine Äquivalenzrelation auf der Menge der geordneten Basen von $V$.
\end{lemma}

