\section{Orientierung}


Im Folgenden sei $V$ ein endlichdimensionaler $\Reals$-Vektorraum.










\subsection{Definition von Orientierung}


[Male Bild aus Fischer.]


Idee: Wollen zwischen positiv und negativ orietierten Basen unterscheiden.


\begin{lemma}\label{lem: possible definition of same orientation}
  Es seien $\basis{B} = (v_1, \dotsc, v_n)$ und $\basis{C} = (w_1, \dotsc, w_n)$ zwei Basen von $V$.
  Dann sind die folgenden Bedingungen äquivalent:
  \begin{enumerate}[leftmargin=*, label=\roman*)]
    \item
      Für den eindeutigen Automorphismus $\Phi \colon V \to V$ mit $\Phi(v_i) = w_i$ für alle $i = 1, \dotsc, n$ gilt $\det(\Phi) > 0$.
    \item
      Für alle alternierende $n$-Linearformen $\omega \colon V^{\times k} \to \Reals$ gilt
      \[
        \omega(v_1, \dotsc, v_n) > 0
        \iff
        \omega(w_1, \dotsc, w_n) > 0.
      \]
  \end{enumerate}
\end{lemma}


\begin{proof}
  Für jede alternierende $n$-Linearform $\omega \colon V^{\times k} \to \Reals$ gilt
  \[
      \omega(w_1, \dotsc, w_n)
    = \omega(\Phi(ve_1), \dotsc, \Phi(v_n))
    = \det(\Phi) \cdot \omega(v_1, \dotsc, v_n).
    \qedhere
  \]
\end{proof}



\begin{definition}
  Zwei geordnete Basen von $V$ heißen \emph{gleichorientiert}, wenn sie die Bedingungen aus Lemma~\ref{lem: possible definition of same orientation} erfüllen.
\end{definition}


\begin{lemma}
  Ist $V$ ein endlichdimensionaler $\Reals$-Vektorraum, so ist Gleichorientiertheit eine Äquivalenzrelation auf der Menge der geordneten Basen von $V$.
\end{lemma}


\begin{definition}\label{def: equivalence relation on alternating n-forms}
  Es sei $A$ der eindimensionale $\Reals$-Vektorraum der alternierenden $n$-Linear\-for\-men $V^{\times n} \to \Reals$. Für alle $\omega_1, \omega_2 \in A \smallsetminus \{0\}$ sei
  \[
    \omega_1 \sim \omega_2
    \iff
    \text{es gibt $\lambda > 0$ mit $\omega_2 = \lambda \omega_1$}.
  \]
\end{definition}


\begin{lemma}
  In der Situation von Definition~\ref{def: equivalence relation on alternating n-forms} definiert $\sim$ eine Äquivalenzrelation.
\end{lemma}


\begin{remark}
  Es gibt nun vier Möglichkeiten eine Orietierung auf $V$ anzugeben:
  \begin{itemize}
    \item
      Man gibt eine geordnete Basis $\basis{B}$ als positiv orientiert vor.
      Eine geordnete Basis $\basis{C}$ von $V$ heißt dann positiv orientiert, wenn sie gleichorientiert zu $\basis{B}$ ist.
    \item
      Man wählt eine Äquivalenzklasse bezüglich der Äquivalenzrelation der Gleichorientiertheit.
      Eine geordnete Basis $\basis{C}$ von $V$ heißt dann positiv orientiert, wenn sie in dieser Äquivalenzklasse enthalten ist.
    \item
      Man gibt eine alternierende $n$-Linearform $\omega \colon V^{\times n} \to \Reals$ vor.
      Eine geordnete Basis $\basis{C} = (v_1, \dotsc, v_n)$ von $V$ heißt dann positiv orientiert, wenn $\omega(v_1, \dotsc, v_n)$.
    \item
      Man gibt eine Äquivalenzklasse $O$ von alternierenden $n$-Linearformen bezüglich $\sim$ an.
      Eine geordnete Basis $\basis{C} = (v_1, \dotsc, v_n)$ von $V$ heißt dann positiv orientiert, wenn $\omega(v_1, \dotsc, v_n) > 0$ für alle $\omega \in O$.
  \end{itemize}
\end{remark}


\begin{definition}
  Eine Orientierung auf $V$ ist eine Äquivalenzklasse bezüglich $\sim$ wie in Definition~\ref{def: equivalence relation on alternating n-forms}.
\end{definition}


\begin{remark}
  Im Fall $V = \Reals^n$ wählt man $d$ meistens durch $d(e_1, \dotsc, e_n) = 1$.
  Dies ist die \emph{Standardorientierung} von $\Reals^n$.
\end{remark}










\subsection{Normiertheit von alternierenden \texorpdfstring{$n$}{n}-Linearformen}


\begin{theorem}
  Es sei $V$ ein $n$-dimensionaler Skalarproduktraum für $\dim V \geq 1$.
  Für jede alternierende $n$-Linearform $d \colon V^{\times n} \to K$ mit $d \neq 0$ gibt es eine eindeutige positive reelle Zahl $c_d$, so dass
  \[
    \left|
      \begin{matrix}
        \bracket{v_1, w_1}  & \cdots  & \bracket{v_1, w_n}  \\
        \vdots              & \ddots  & \vdots              \\
        \bracket{v_n, w_1}  & \cdots  & \bracket{v_n, w_n}
      \end{matrix}
    \right|
    =
    c_d d(v_1, \dotsc, v_n) \overline{ d(w_1, \dotsc, w_n) }
  \]
  für alle $v_1, \dotsc, v_n, w_1, \dotsc, w_n \in V$.
\end{theorem}


\begin{definition}
  Eine alternierende $n$-Linearform auf einem $n$-dimensionalen Skalarproduktraum $V \neq 0$ heißt \emph{normiert}, falls $c_d = 1$.
\end{definition}


\begin{lemma}
  Ist $V \neq 0$ ein $n$-dimensionaler Skalarproduktraum und $d \colon V^{\times n} \to \Korper$ eine alternierende $n$-Linearform, dann sind die folgenden Bedingungen äquivalent:
  \begin{enumerate}[leftmargin=*, label=\roman*)]
    \item
      $d$ ist normiert.
    \item
      Es gibt eine Orthonormalbasis $\basis{B} = (v_1, \dotsc, v_n)$ von $V$ mit $|d(v_1, \dotsc, v_n)| = 1$.
    \item
      Füre jede Orthonormalbasis $\basis{B} = (v_1, \dotsc, v_n)$ von $V$ gilt $|d(v_1, \dotsc, v_n)| = 1$.
  \end{enumerate}
\end{lemma}


\begin{proof}
  Für jede Orthonormalbasis $\basis{B} = (v_1, \dotsc, v_n)$ von $V$ gilt $c_d = 1/|d(v_1, \dotsc, v_n)|^2$.
\end{proof}


\begin{lemma}
  Es sei $V \neq 0$ ein Skalarproduktraum und $\omega_1 \colon V^{\times n} \to \Korper$ eine normierte alternierende $n$-Linearform.
  Dann ist eine alternierende $n$-Linearform $\omega_2 \colon V^{\times n} \to \Korper$ genau dann normiert, wenn es ein $\lambda \in \Korper$ mit $|\lambda| = 1$ und $\omega_1 = \lambda \omega_2$ gibt.
\end{lemma}










\subsection{Normierte, alternierende \texorpdfstring{$n$}-Linearformen}


Im Folgenden sei $V \neq 0$ ein euklidischer Vektorraum und $d \colon V^{\times n} \to \Reals$ eine normiertie, alternierende $n$-Linearform.
Indem wir diese als positiv auszeichnen, erhalten wir eine Orientierung auf $V$.


\begin{definition}
  Es sei $\dim V = 2$.
  Für $v_1, v_2 \in V$ mit $v_1, v_2 \neq 0$  ist der unorientierte Winkel zwischen $v_1, v_2$ das eindeutige Element $\theta \in [0,2\pi)$ mit
  \[
    \cos(\theta) = \frac{\bracket{v,w}}{\|v\| \|w\|}
    \quad\text{und}\quad
    \sin(\theta) = \frac{d(v,w)}{\|v\| \|w\|}.
  \]
\end{definition}


\begin{lemma}\label{lemma: existince of vector product}
  Es sei $V$ ein dreidimensionaler orientierter, euklidischer Vektorraum.
  Ist $d$ die eindeutige positive, normierte alternierende $n$-Linearform auf $V$, so gibt es für alle $v_1, v_2 \in V$ ein eindeutiges Element $v_1 \times v_2 \in V$ mit
  \[
    \bracket{v_1 \times v_2, w} = d(v_1, v_2, w)
    \quad
    \text{für alle $w \in V$}.
  \]
\end{lemma}


\begin{definition}
  In der Situation von Lemma~\ref{lemma: existince of vector product} heißt $v_1 \times v_2$ das \emph{Vektorprodukt von $v_1$ und $v_2$}.
\end{definition}


\begin{theorem}[Eigenschaften des Vektorprodukts]
  Es sei $V$ wie in Lemma~\ref{lemma: existince of vector product}.
  \begin{enumerate}[leftmargin=*, label=\roman*)]
    \item
      Die Abbildung $V \times V \to V$, $(v_1, v_2) \mapsto v_1 \times v_2$ ist bilinear und alternierend.
    \item
      Ist $(v_1, v_2, v_3)$ eine Orthonormalbasis von $V$, so gilt
      \[
        v_1 \times v_2 =  v_3,
        \quad
        v_1 \times v_3 = -v_2,
        \quad
        v_2 \times v_3 = v_3.
      \]
      Allgemeiner gilt
      \begin{align*}
         &\,  (a_1 v_1 + a_2 v_2 + a_3 v_3) \times (b_1 v_1 + b_2 v_2 + b_3 v_3)          \\
        =&\,  (a_2 b_3 - a_3 b_2) v_1 + (a_3 b_1 - a_1 b_3) v_2 + (a_1 b_2 - a_2 b_1) v_3.
      \end{align*}
    \item
      Es gilt die Jacobi-Identität
      \[
        u \times (v \times w) + v \times (u \times w) + w \times (u \times v) = 0
        \quad
        \text{für alle $u, v, w \in V$}.
      \]
    \item
      Es gilt
      \[
          \bracket{v_1 \times v_2, w_1 \times w_2}
        = \bracket{v_1, w_1} \bracket{v_2, w_2} - \bracket{v_1, w_2} \bracket{v_2, w_1}.
      \]
    \item
      Es gilt
      \[
          u \times (v \times w)
        = \bracket{u, w} v - \bracket{u, v} w.
      \]
    \item
      Ist $\theta$ der unorienierte Winkel zwischen $v_1, v_2 \in V$ mit $v_1, v_2 \neq 0$, so gilt
      \[
        \|v_1 \times v_2\| = \|v_1\| \|v_2\| \sin(\theta).
      \]
  \end{enumerate}
\end{theorem}

























