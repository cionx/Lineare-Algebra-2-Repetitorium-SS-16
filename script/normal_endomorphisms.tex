\section{Normale Endomorphismen}


Im Folgenden seien alle Vektorräume endlichdimensional, sofern nicht anders angegeben.










\subsection{Grundlegende Definitionen und Eigenschaften}


\begin{proposition}\label{prop: existence and uniqueness of adjoints}
  Für zwei Skalarprodukträume $V$ und $W$ gibt es für jede $\Korper$-lineare Abbildung $f \colon V \to W$ eine eindeutige $\Korper$-lineare Abbildung $g \colon W \to V$ mit
  \[
    \bracket{f(v), w} = \bracket{v, g(w)}
    \quad
    \text{für alle $v \in V$ und $w \in W$}.
  \]
\end{proposition}


\begin{definition}
  In der Situation von Proposition~\ref{prop: existence and uniqueness of adjoints} ist die Abbildung $g$ die zu $f$ \emph{adjungierte} Abbildung, und wird mit $f^*$ notiert.
\end{definition}


\begin{proposition}
  Es seien $U, V, W$ drei Skalarprodukträume.
  \begin{enumerate}[leftmargin=*, label=\roman*)]
    \item
      Es gilt $\id_V^* = \id_V$, und für alle linearen Abbildungen $f \colon V \to W$ und $g \colon U \to V$ gilt
      \[
        (f g)^* = g^* f^*.
      \]
    \item
      Ist $f \colon V \to W$ ein Isomorphismus, so ist auch $f^*$ ein Isomorphismus.
    \item
      Für alle $\Korper$-linearen Abbildungen $f, g \colon V \to W$ und $\lambda \in \Korper$ gilt
      \begin{enumerate}[leftmargin=*, label=\alph*)]
        \item
          $(f^*)^* = f$,
        \item
          $(f + g)^* = f^* + g^*$ und
        \item
          $(\lambda f)^* = \overline{\lambda} f^*$.
      \end{enumerate}
      Insbesondere ist die Abbildung $\Hom_\Korper(V,W) \to \Hom_\Korper(W,V)$, $f \mapsto f^*$ ein Isomorphismus (bzw.\ Antiisomorphismus) von $\Reals$-Vektorräumen (bzw.\ $\Complex$-Vektorräumen).
  \end{enumerate}
\end{proposition}


\begin{definition}
  Für $A \in \Mat(m \times n, \Korper)$ ist $A^* \coloneqq \overline{A^t} = (\overline{A})^t \in \Mat(n \times m, \Korper)$.
\end{definition}


\begin{proposition}
  Es sei $V$ ein Skalarproduktraum mit geordneter Orthonormalbasis $\basis{B}$ und $W$ ein Skalarproduktraum mit geordneter Orthonormalbasis $\basis{C}$.
  Für jede $\Korper$-lineare Abbildung $f \colon V \to W$ gilt dann
  \[
    \Mat_{\basis{C}, \basis{B}}(f^*) = \Mat_{\basis{B}, \basis{C}}(f)^*.
  \]
\end{proposition}


\begin{definition}
  Ein Endomorphismus $f \colon V \to V$ eines Skalarproduktraums $V$ heißt
  \begin{enumerate}[leftmargin=*, label=\roman*)]
    \item
      \emph{normal}, falls $f$ und $f^*$ kommutieren (also $f f^* = f^* f$),
    \item
      \emph{selbstadjungiert}, falls $f = f^*$,
    \item
      \emph{antiselbstadjungiert}, falls $f^* = -f$,
    \item
      \emph{orthogonal} (für $\Korper = \Reals$), bzw.\ \emph{unitär} (für $\Korper = \Complex$) falls $f$ ein Isomorphismus mit $f^* = f^{-1}$ ist.
  \end{enumerate}
  Für den Fall $\Korper = \Complex$ sei
  \[
    \Unitary(V)
    \coloneqq
    \{
      f \colon V \to V
      \mid
      \text{$f$ ist unitär}
    \}
  \]
  und für den Fall $\Korper = \Reals$ sei
  \[
    \Orthogonal(V)
    \coloneqq
    \{
      f \colon V \to V
      \mid
      \text{$f$ ist orthogonal}.
    \}
  \]
  Eine Matrix $A \in \Mat_n(\Korper)$ heißt
  \begin{enumerate}[leftmargin=*, label=\Roman*)]
    \item
      \emph{normal}, falls $A$ und $A^*$ kommutieren (also $A A^* = A^* A$),
    \item
      \emph{selbstadjungiert}, falls $A = A^*$,
    \item
      \emph{antiselbstadjungiert}, falls $A^* = -A$,
    \item
      \emph{orthogonal} (für $\Korper = \Reals$), bzw.\ \emph{unitär} (für $\Korper = \Complex$) falls $A$ invertierbar ist und $A^* = A^{-1}$.
  \end{enumerate}
  Für alle $n \geq 1$ seien
  \[
    \Orthogonal(n)
    \coloneqq
    \{
      O \in \Mat_n(\Reals)
      \mid
      \text{$O$ ist orthogonal}
    \}
    \quad\text{und}\quad
    \Unitary(n)
    \coloneqq
    \{
      U \in \Mat_n(\Complex)
      \mid
      \text{$U$ ist unitär}
    \}.
  \]
\end{definition}


\begin{proposition}
  Es sei $f \colon V \to V$ ein Endomorphismus eines Skalarproduktraums $V$, und $\basis{B}$ eine geordnete Basis von $V$.
  Dann ist $f$ genau dann normal, selbstadjungiert, antiselbstadjungiert, orthogonal, bzw.\ unitär, wenn $\Mat_\basis{B}(f)$ normal, selbstadjungiert, antiselbstadjungiert, orthogonal, bzw.\ unitär ist.
\end{proposition}


\begin{proposition}
  Es sei $f \colon V \to V$ ein normaler Endomorphismus eines Skalarproduktraums $V$.
  \begin{enumerate}[leftmargin=*, label=\roman*)]
    \item
      Für alle $v \in V$ ist $\|f(v)\| = \|f^*(v)\|$.
    \item
      Es ist $V_\lambda(f^*) = V_{\overline{\lambda}}(f)$ für alle $\lambda \in \Korper$.
    \item
      Für alle $\lambda, \mu \in \Korper$ mit $\lambda \neq 0$ ist $V_\lambda(f) \perp V_\mu(f)$.
    \item
      Für $v \in V$ und $n \geq 1$ mit $f^n(v) = 0$ ist bereits $f(v) = 0$.
    \item
      Für alle $\lambda \in V$ ist $V^\sim_\lambda(f) = V_\lambda(f)$.
    \item
      Es ist $\im f^* = (\ker f)^\perp$ und $\ker f^* = (\im f)^\perp$.
    \item
      Ein Untervektorraum $U \subseteq V$ ist genau dann $f$-invariant, wenn $U^\perp$ invariant unter $f^*$ ist.
  \end{enumerate}
\end{proposition}


\begin{proposition}
  Für einen Endomorphismus $f \colon V \to V$ eines Skalarproduktraums $V$ sind die folgenden Bedingungen äquivalent:
  \begin{enumerate}[leftmargin=*, label=\roman*)]
    \item
      $f$ ist orthogonal (für $\Korper = \Reals$), bzw.\ unitär (für $\Korper = \Complex$), d.h.\ $f$ ist ein Isomorphismus mit $f^* = f^{-1}$.
    \item
      Es ist $f f^* = \id_V$.
    \item
      Es ist $f^* f = \id_V$.
    \item
      Für alle $v \in V$ ist $\|f(v)\| = \|v\|$.
    \item
      Für alle $v, w \in V$ ist $\bracket{f(v), f(w)} = \bracket{v,w}$.
  \end{enumerate}
\end{proposition}


\begin{corollary}
  Für jede Matrix $A \in \Mat_n(\Korper)$ sind die folgenden Bedingungen äquivalent:
  \begin{enumerate}[leftmargin=*, label=\roman*)]
    \item
      $A$ ist orthogonal (für $\Korper = \Reals$), bzw.\ unitär (für $\Korper = \Complex$), d.h.\ $A$ ist invertierbar mit $A^* = A^{-1}$.
    \item
      Es ist $A A^* = I$.
    \item
      Es ist $A^* A = I$.
    \item
      Für alle $x \in \Korper^n$ ist $\|Ax\| = \|x\|$.
    \item
      Für alle $x, y \in \Korper^n$ ist $\bracket{Ax, Ay} = \bracket{x, y}$.
    \item
      Die Spalten von $A$ sind eine Orthonormalbasis von $\Korper^n$.
    \item
      Die Zeilen von $A$ sind eine Orthonormalbasis von $\Korper^n$.
  \end{enumerate}
\end{corollary}










\subsection{Normalenformen für \texorpdfstring{$\Korper = \Complex$}{K = C}}



\begin{theorem}
  Es sei $f \colon V \to V$ ein Endomorphismus eines unitären Vektorraums $V$.
  Die folgenden Bedingungen sind äquivalent:
  \begin{enumerate}[leftmargin=*, label=\roman*)]
    \item
      $f$ ist normal.
    \item
      $V$ hat eine Orthonormalbasis aus Eigenvektoren von $f$.
    \item
      Für jeden $f$-invarianten Untervektorraum $U \subseteq V$ ist auch $U^\perp$ invariant unter $f$.
  \end{enumerate}
\end{theorem}


\begin{corollary}
  Für $A \in \Mat_n(\Complex)$ sind die folgenden Bedingungen äquivalent:
  \begin{enumerate}[leftmargin=*, label=\roman*)]
    \item
      $A$ ist normal.
    \item
      Es gibt eine unitäre Matrix $U \in \Unitary(n)$, so dass $U\!A U^{-1}$ in Diagonalgestalt ist.
  \end{enumerate}
\end{corollary}


\begin{proposition}
  Es sei $f \colon V \to V$ ein Endomorphismus eines unitären Vektorraums $V$.
  \begin{enumerate}[leftmargin=*, label=\roman*)]
    \item
      $f$ ist genau dann selbstadjungiert, wenn $f$ normal mit reellen Eigenwerten ist.
    \item
      $f$ ist genau dann antiselbstadjungiert, wenn $f$ normal mit rein imaginären Eigenwerten ist.
    \item
      $f$ ist genau dann unitär, wenn $f$ normal ist und alle Eigenwerte Betrag $1$ haben.
  \end{enumerate}
\end{proposition}


\begin{corollary}
  Es sei $A \in \Mat_n(\Complex)$.
  \begin{enumerate}[leftmargin=*, label=\roman*)]
    \item
      $A$ ist genau dann selbstadjungiert, wenn es $U \in \Unitary(n)$ gibt, so dass $U\!A U^{-1}$ eine Diagonalmatrix mit reellen Diagonaleinträgen ist.
    \item
      $A$ ist genau dann antiselbstadjungiert, wenn es $U \in \Unitary(n)$ gibt, so dass $U\!A U^{-1}$ eine Diagonalmatrix mit rein imaginären Diagonaleinträgen ist.
    \item
      $A$ ist genau dann unitär, wenn es $U \in \Unitary(n)$ gibt, so dass $U\!A U^{-1}$ eine Diagonalmatrix ist, deren Diagonaleinträge alle Betrag $1$ haben.
  \end{enumerate}
\end{corollary}


\begin{corollary}
  Für alle $n \geq 1$ ist $\det(\Unitary(n)) = S^1$, wobei $S^1 \coloneqq \{z \in \Complex \mid |z| = 1\}$.
\end{corollary}










\subsection{Normalenformen für \texorpdfstring{$\Korper = \Reals$}{K = R}}


\begin{notation}
  Für alle $\varphi \in \Reals$ sei
  \[
    D(\varphi)
    \coloneqq
    \begin{pmatrix}
      \cos(\varphi) &          - \sin(\varphi)  \\
      \sin(\varphi) & \phantom{-}\cos(\varphi)
    \end{pmatrix}
    \in
    \Mat_2(\Reals).
  \]
\end{notation}


\begin{theorem}\label{thrm: normal form for real normal endomorphisms}
  Für einen Endomorphismus $f \colon V \to V$ eines euklidischen Vektorraum $V$ sind die folgenden Bedingungen äquivalent:
  \begin{enumerate}[leftmargin=*, label=\roman*)]
    \item
      $f$ ist normal.
    \item
      Es gibt eine geordnete Orthonormalbasis $\basis{B}$ von $V$, so dass
      \[
        \Mat_\basis{B}(f)
        =
        \begin{pmatrix}
          \lambda_1 &         &           &                   &         &                   \\
                    & \ddots  &           &                   &         &                   \\
                    &         & \lambda_s &                   &         &                   \\
                    &         &           & r_1 D(\varphi_1)  &         &                   \\
                    &         &           &                   & \ddots  &                   \\
                    &         &           &                   &         & r_t D(\varphi_t)
        \end{pmatrix}
      \]
      mit $\lambda_1, \dotsc, \lambda_s \in \Reals$, $r_1, \dotsc, r_t > 0$ und $\varphi_1, \dotsc, \varphi_t \in (0, \pi)$.
  \end{enumerate}
  Dabei sind $\lambda_1, \dotsc, \lambda_s$ die Eigenwerte von $f$, und die Paare $(r_1, \varphi_1), \dotsc, (r_t, \varphi_t)$ sind eindeutig bis auf Permutation.
\end{theorem}


\begin{corollary}\label{cor: normal form for real normal matrices}
  Für einen Matrix $A \in \Mat_n(\Reals)$ sind die folgenden Bedingungen äquivalent:
  \begin{enumerate}[leftmargin=*, label=\roman*)]
    \item
      $A$ ist normal.
    \item
      Es gibt eine Matrix $O \in \Orthogonal(n)$, so dass 
      \[
        O A O^{-1}
        =
        \begin{pmatrix}
          \lambda_1 &         &           &                   &         &                   \\
                    & \ddots  &           &                   &         &                   \\
                    &         & \lambda_s &                   &         &                   \\
                    &         &           & r_1 D(\varphi_1)  &         &                   \\
                    &         &           &                   & \ddots  &                   \\
                    &         &           &                   &         & r_t D(\varphi_t)
        \end{pmatrix}
      \]
      mit $\lambda_1, \dotsc, \lambda_s \in \Reals$, $r_1, \dotsc, r_t > 0$ und $\varphi_1, \dotsc, \varphi_t \in (0, \pi)$.
  \end{enumerate}
  Dabei sind $\lambda_1, \dotsc, \lambda_s$ die Eigenwerte von $A$, und die Paare $(r_1, \varphi_1), \dotsc, (r_t, \varphi_t)$ sind eindeutig bis auf Permutation.
\end{corollary}


\begin{proposition}
  Es sei $f \colon V \to V$ ein Endomorphismus eines euklidischen Vektorraums $V$.
  \begin{enumerate}[leftmargin=*, label=\roman*)]
    \item
      $f$ ist genau dann selbstadjungiert, wenn $f$ normal ist und in der Normalenform von Theorem~\ref{thrm: normal form for real normal endomorphisms} $t = 0$ gilt, also wenn $V$ eine Orthonormalbasis aus Eigenvektoren von $f$ besitzt.
    \item
      $f$ ist genau dann antiselbstadjungiert, wenn $f$ normal ist und in der Normalenform von Theorem~\ref{thrm: normal form for real normal endomorphisms} $\lambda_1 = \dotsb = \lambda_s = 0$ und $\varphi_1 = \dotsb = \varphi_t = \pi/2$ gilt.
    \item
      $f$ ist genau dann orthogonal, wenn $f$ normal ist und in der Normalenform von Theorem~\ref{thrm: normal form for real normal endomorphisms} $\lambda_i = \pm 1$ für alle $i = 1, \dotsc, s$ und $r_1 = \dotsb = r_t = 1$ gilt.
  \end{enumerate}
\end{proposition}


\begin{corollary}
  Es sei $A \in \Mat_n(\Reals)$.
  \begin{enumerate}[leftmargin=*, label=\roman*)]
    \item
      $A$ ist genau dann selbstadjungiert, wenn $A$ normal ist, und in der Normalenform von Korollar~\ref{cor: normal form for real normal matrices} $t = 0$ gilt, also wenn es $O \in \Orthogonal(n)$ gibt, so dass $O A O^{-1}$ in Diagonalgestalt ist.
    \item
      $A$ ist genau dann antiselbstadjungiert, wenn $A$ normal ist und in der Normalenform von Korollar~\ref{cor: normal form for real normal matrices} $\lambda_1 = \dotsb = \lambda_s = 0$ und $\varphi_1 = \dotsb = \varphi_t = \pi/2$ gilt.
    \item
      $A$ ist genau dann orthogonal, wenn $A$ normal ist und in der Normalenform von Korollar~\ref{cor: normal form for real normal matrices} $\lambda_i = \pm 1$ für alle $i = 1, \dotsc, s$ und $r_1 = \dotsb = r_t = 1$ gilt.
  \end{enumerate}
\end{corollary}


\begin{corollary}
  Für alle $n \geq 1$ ist $\det(\Orthogonal(n)) = \{1,-1\}$.
\end{corollary}










\subsection{Orthogonale Projektionen}


\begin{lemma}\label{lem: existence and uniqueness of projections}
  Sind $U, W \subseteq V$ Untervektorräume eines $K$-Vektorraums $V$ mit $V = U \oplus W$, so gibt es genau eine lineare Abbildung $P \colon V \to V$ mit
  \[
    P(u + w) = u
    \quad
    \text{für alle $u \in U$, $w \in W$}.
  \]
\end{lemma}


\begin{definition}
  \begin{enumerate}[leftmargin=*, label=\roman*)]
    \item
      In der Situation von Lemma~\ref{lem: existence and uniqueness of projections} heißt $P$ die \emph{Projektion} auf $U$ entlang $W$.
    \item
      Ist $U \subseteq V$ ein Untervektorraum eines Skalarproduktraums $V$, so heißt die Projektion auf $U$ entlang $U^\perp$ die \emph{orthogonale Projektion} auf $U$.
  \end{enumerate}
\end{definition}


\begin{proposition}
  \begin{enumerate}[leftmargin=*, label=\roman*)]
    \item
      Für einen Endomorphismus $P \colon V \to V$ eines $K$-Vektorraums $V$ sind die folgenden Bedingungen äquivalent:
      \begin{enumerate}[leftmargin=*, label=\alph*)]
        \item
          Es gilt $P^2 = P$.
        \item
          $P$ ist die Projektion auf $\im(P)$ entlang $\ker(P)$.
      \end{enumerate}
    \item
      Für einen Endomorphismus $P \colon V \to V$ eines Skalarproduktraums $V$ sind die folgenden Bedingungen äquivalent:
      \begin{enumerate}[leftmargin=*, label=\alph*)]
        \item
          Es gilt $P^2 = P$ und $P$ ist normal.
        \item
          $P$ ist die orthogonale Projektion auf $\im(P)$.
      \end{enumerate}
  \end{enumerate}
\end{proposition}












\subsection{Polarzerlegung}


\begin{definition}
  \begin{enumerate}[leftmargin=*, label=\roman*)]
    \item
      Ein selbstadjungierter Endomorphismus $f \colon V \to V$ eines Skalarproduktraums $V$ heißt \emph{positiv}, wenn alle Eigenwerte von $f$ positiv sind.
    \item
      Eine selbtadjungierte Matrix $A \in \Mat_n(\Korper)$ heißt positiv, wenn alle Eigenwerte von $A$ positiv sind.
  \end{enumerate}
\end{definition}


\begin{proposition}
  \begin{enumerate}
    \item
      Ist $f \colon V \to V$ ein positiver selbstadjungierter Endomorphismus, so gibt es einen eindeutigen positiven selbstadjungierten Endomorphismus $g \colon V \to V$ mit $f = g^2$.
    \item
      Ist $A \in \Mat_n(\Korper)$ positiv selbstadjungiert, so gibt es eine eindeutige positive selbstadjungierte Matrix $B \in \Mat_n(\Korper)$ mit $A = B^2$.
  \end{enumerate}
\end{proposition}


\begin{theorem}
  \begin{enumerate}[leftmargin=*, label=\roman*)]
    \item
      Ist $f \in \GL(V)$ für einen Skalarproduktraum, so gibt es eindeutige positive selbstadjungierte Endomorphismen $g_1, g_2 \colon V \to V$ und eindeutige $s_1, s_2 \in \Unitary(V)$ (für $\Korper = \Complex$), bzw.\ $s_1, s_2 \in \Orthogonal(V)$ (für $\Korper = \Reals$) mit
      \[
        f = s_1 g_1 = g_2 s_2.
      \]
  \end{enumerate}
\end{theorem}




























