\section{Simultane Diagonalisierbarkeit}


\begin{lemma}\label{lem: eigensummands are in the subspace}
  Es sei $f \colon V \to V$ ein Endomorphismus eines $K$-Vektorraums $V$ und \mbox{$U \subseteq V$} ein $f$-invarianter Untervektorraum.
  Ist $u \in U$ mit $u = v_1 + \dotsb + v_n$, wobei $v_i \in V_{\lambda_i}(f)$ für alle $i = 1, \dotsc, n$ und $\lambda_1, \dotsc, \lambda_n \in K$ paarweise verschieden sind, so ist bereits $v_1, \dotsc, v_n \in V$.
\end{lemma}


\begin{proof}
  Wir zeigen die Aussage per Induktion über $n$.
  Für $n = 1$ gilt $v_1 = u \in U$.
  Ist $n \geq 2$, so gilt
  \[
      f(u)
    = f(v_1) + \dotsb + f(v_n)
    =  \lambda_1 v_1 + \dotsb + \lambda_n v_n
  \]
  und somit
  \begin{align*}
    U \ni \lambda_1 u - f(u)
      &=  \lambda_1 (v_1 + \dotsb + v_n) - (\lambda_1 v_1 + \dotsb + \lambda_n v_n) \\
      &=  (\lambda_1 - \lambda_2) v_2 + \dotsb + (\lambda_1 - \lambda_n) v_n.
  \end{align*}
  Da $(\lambda_1 - \lambda_i) v_i \in V_{\lambda_i}(f)$ für alle $i = 1, \dotsc, n$ ergibt sich per Induktionsvoraussetzung, dass $(\lambda_1 - \lambda_i) v_i \in U$ für alle $i = 2, \dotsc, n$.
  Da $\lambda_1 - \lambda_i \neq 0$ für alle $i = 2, \dotsc, n$ (denn die $\lambda_j$ sind paarweise verschieden) gilt $v_2, \dotsc, v_n \in U$.
  Somit gilt auch \mbox{$v_1 = u - v_2 - \dotsb - v_n \in U$}.
\end{proof}


\begin{lemma}\label{lem: directness of sum of eigenspaces}
  Ist $f \colon V \to V$ ein Endomorphismus eines $K$-Vektorraums $V$, so ist die Summe $\sum_{\lambda \in K} V_\lambda(f)$ direkt.
\end{lemma}


\begin{proof}
  Es seien $\lambda_1, \dotsc, \lambda_n \in K$ paarweise verschieden und $v_i \in V_{\lambda_i}(f)$ für $i = 1, \dotsc, n$, so dass $0 = v_1 + \dotsb + v_n$.
  Anwenden von Lemma~\ref{lem: eigensummands are in the subspace} auf den Untervektorraum $U = 0$ ergibt, dass $v_1, \dotsc, v_n \in 0$ und somit $v_1, \dotsc, v_n = 0$.
\end{proof}


\begin{definition}
  Ein Endomorphismus $f \colon V \to V$ eines $K$-Vektorraums $V$ heißt \emph{diagonalisierbar}, falls $V = \bigoplus_{\lambda \in K} V_\lambda$.
\end{definition}


\begin{remark}
  Es sei $f \colon V \to V$ ein Endomorphismus eines $K$-Vektorraums $V$.
  \begin{enumerate}[leftmargin=*, label=\roman*)]
    \item
      Nach Lemma~\ref{lem: directness of sum of eigenspaces} ist $f$ genau dann diagonalisierbar, wenn $V = \sum_{\lambda \in K} V_\lambda(f)$.
      Es genügt also, jeden Vektor als Summe, bzw.\ Linearkombination von Eigenvektoren zu schreiben.
    \item
      Ist $V$ endlichdimensional, so ist $f$ genau dann diagonalisierbar, wenn $V$ eine Basis aus Eigenvektoren von $f$ besitzt.
      
      Ist nämlich $V = \bigoplus_{\lambda \in K} V_\lambda(f)$ und sind $\lambda_1, \dotsc, \lambda_n \in K$ die Eigenwerte von $f$, so ist $V_\mu(f) = 0$ für alle $\mu \in K \smallsetminus \{\lambda_1, \dotsc, \lambda_n\}$ und somit $V = V_{\lambda_1}(f) \oplus \dotsb \oplus V_{\lambda_n}(f)$.
      Wählt man nun von jeder dieser endlich vielen Eigenräume eine Basis, so ergibt sich durch Zusammenfügen dieser Basen eine Basis von $V$, die aus Eigenvektoren besteht.
      
      Hat andererseits $V$ eine Basis aus Eigenvektoren, so ist jeder Vektor $v \in V$ eine Linearkombination von Eigenvektoren und somit $V = \sum_{\lambda \in K} V_\lambda(f)$.
  \end{enumerate}
\end{remark}


\begin{proposition}
  Ist $f \colon V \to V$ ein diagonalisierbarer Endomorphismus eines $K$-Vek\-tor\-raums $V$, so ist für jeden $f$-invarianten Untervektorraum $U \subseteq V$ auch die Einschränkung $f|_U \colon U \to U$ diagonalisierbar, und es gilt
  \[
    U = \bigoplus_{\lambda \in K} [ U \cap V_\lambda(f) ].
  \]
\end{proposition}


\begin{proof}
  Es sei $u \in U$.
  Da $f$ diagonalisierbar ist, gibt es paarweise verschiedene Eigenwerte $\lambda_1, \dotsc, \lambda_n \in K$ und $v_i \in V_{\lambda_i}(f)$ mit $u = v_1 + \dotsb + v_n$.
  Nach Lemma~\ref{lem: eigensummands are in the subspace} gilt bereits $v_1, \dotsc, v_n \in U$.
  Da damit $v_i \in U \cap V_{\lambda_i}(f) = U_{\lambda_i}(f|_U)$ für alle $i = 1, \dotsc, n$ gilt, ergibt sich $u \in \sum_{i=1}^n U_{\lambda_i}(f|_U) = \sum_{\lambda \in K} U_{\lambda}(f|_U)$.
  Damit ist $U = \sum_{\lambda} U_\lambda(f)$, und somit \mbox{$U = \bigoplus_{\lambda} U_\lambda(f|_U)$}.
  Die letzte Gleichung ergibt sich mit $U_\lambda(f|_U) = U \cap V_\lambda(f)$.
\end{proof}


\begin{definition}
  Eine Kollektion von Endomorphismen $f_1, \dotsc, f_n \colon V \to V$ eines $K$-Vek\-tor\-raums $V$ heißt \emph{simultan diagonalisierbar}, falls
  \[
    V
    = \bigoplus_{\lambda_1, \dotsc, \lambda_n \in K}
      \bigg(
        \underbrace{ V_{\lambda_1}(f_1) \cap \dotsb \cap V_{\lambda_n}(f_n) }_{\text{gemeinsame Eigenvektoren}}
      \bigg).
  \]
\end{definition}


\begin{remark}
  Ist $V$ endlichdimensional, so sind $f_1, \dotsc, f_n \colon V \to V$ genau dann simultan diagonalisierbar, wenn $V$ eine Basis aus gemeinsamen Eigenvektoren von $f_1, \dotsc, f_n$ besitzt.
\end{remark}


\begin{theorem}\label{thrm: simultaneous diagonalizability}
  Eine Kollektion von Endomorphismes $f_1, \dotsc, f_n \colon V \to V$ eines $K$-Vek\-tor\-raums $V$ ist genau dann simultan diagonalisierbar, wenn die Endomorphismen kommutieren und jeweils einzeln diagonalisierbar sind.
\end{theorem}


\begin{lemma}
  Sind $f, g \colon V \to V$ zwei kommutierende Endomorphismen eines $K$-Vek\-tor\-raums $V$, so ist $V_\lambda(f)$ für alle $\lambda \in K$ invariant unter $g$.
\end{lemma}


\begin{proof}[Beweis von Theorem~\ref{thrm: simultaneous diagonalizability}]
  Ist $v \in V$ ein gemeinsamer Eigenvektor von $f_1, \dotsc, f_n$, so gilt $f_i(f_j(v)) = f_j(f_i(v))$ für alle $i, j = 1, \dotsc, n$, denn ist $f_i(v) = \lambda_i v$, so gilt
  \[
    f_i(f_j(v))
    = f_i(\lambda_j v)
    = \lambda_j f_i(v)
    = \lambda_j \lambda_i v
    = \lambda_i \lambda_j v
    = \lambda_i f_j(v)
    = f_j(\lambda_i v)
    = f_j(f_i(v)).
  \]
  Sind $f_1, \dotsc, f_n$ simultan diagonalisierbar, so ist jeder Vektor $v \in V$ die Summe von gemeinsamen Eigenvektoren $v_1, \dotsc, v_n \in V$, also $v = v_1 + \dotsb + v_n$, weshalb
  \[
    f_i(f_j(v))
    = f_i(f_j(v_1)) + \dotsb + f_i(f_j(v_n))
    = f_j(f_i(v_1)) + \dotsb + f_j(f_i(v_n))
    = f_j(f_i(v)).
  \]
  Also gilt dann $f_i f_j = f_j f_i$ für alle $i, j = 1, \dotsc, n$.

  Umgekehrt seien nun $f_1, \dotsc, f_n$ kommutierend und einzeln diagonalisierbar.
  Wir zeigen per Induktion über $n$, dass $f_1, \dotsc, f_n$ simultan diagonalisierbar sind.
  Für $n = 1$ ist nichts zu zeigen.
  
  Es sei nun $n \geq 2$.
  Da $f_1$ diagonalisierbar ist, gilt $V = \bigoplus_{\lambda \in K} V_\lambda(f_1)$.
  Da $f_1, \dotsc, f_n$ kommutieren ist $V_\lambda(f_1)$ für alle $\lambda \in K$ invariant unter $f_2, \dotsc, f_n$.
  Da $f_2, \dotsc, f_n$ diagonalisierbar sind, sind es auch die Einschränkungen $f_i|_{V_\lambda(f_1)}$ für alle $i = 2, \dotsc, n$ und $\lambda \in K$.
  Da $f_2, \dotsc, f_n$ kommutieren, kommutieren auch diese Einschränkungen.
  Per Induktionsvorraussetzung gilt daher
  \begin{align*}
        V_\lambda(f_1)
    &=  \bigoplus_{\lambda_2, \dotsc, \lambda_n}
        \bigg(
          ( V_\lambda(f_1) )_{\lambda_2}(f_2|_{V_\lambda(f_1)})
          \cap \dotsb \cap
          ( V_\lambda(f_1) )_{\lambda_n}(f_n|_{V_\lambda(f_1)})
        \bigg)
    \\
    &=  \bigoplus_{\lambda_2, \dotsc, \lambda_n}
        \bigg(
          V_\lambda(f_1) \cap V_{\lambda_2}(f_2) \cap V_{\lambda_n}(f_n)
        \bigg).
  \end{align*}
  Somit gilt
  \begin{align*}
      V
    =   \bigoplus_{\lambda_1 \in K} V_{\lambda_1}(f_1)
    &=  \bigoplus_{\lambda_1 \in K} \bigoplus_{\lambda_2, \dotsc, \lambda_n \in K}
        \bigg(
          V_{\lambda_1}(f_1) \cap V_{\lambda_2}(f_2) \cap V_{\lambda_n}(f_n)
        \bigg)
    \\
    &=  \bigoplus_{\lambda_1, \dotsc, \lambda_n \in K}
        \bigg(
          V_{\lambda_1}(f_1) \cap \dotsb \cap V_{\lambda_n}(f_n)
        \bigg).
  \end{align*}
\end{proof}







