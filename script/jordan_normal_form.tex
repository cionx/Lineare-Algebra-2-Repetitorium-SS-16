\section{Jordannormalform}










\subsection{Nilpotente Endomorphismus}


\begin{definition}
  Ein Endomorphismus $f \colon V \to V$ eines $K$-Vektorraums $V$ heißt \emph{nilpotent}, falls es ein $n \in \Naturals$ mit $f^n = 0$ gibt.
  Eine Matrix $A \in \Mat_n(K)$ heißt nilpotent, falls es ein $n \in \Naturals$ mit $A^n = 0$ gibt.
\end{definition}


\begin{lemma}
  Ist $f \colon V \to V$ ein Endomorphismus eines endlichdimensionalen $K$-Vek\-tor\-raums $V$, so ist $f$ genau dann nilpotent, wenn für jede geordnete Basis $\basis{B}$ von $V$ die Matrix $\Mat_\basis{B}(f)$ nilpotent ist.
\end{lemma}


\begin{notation}
  Für alle $n \geq 1$ sei
  \[
    J_n
    \coloneqq
    \begin{pmatrix}
      0 & 1       &         &   \\
        & \ddots  & \ddots  &   \\
        &         & \ddots  & 1 \\
        &         &         & 0
    \end{pmatrix}
    \in \Mat_n(K).
  \]
\end{notation}


\begin{theorem}\label{thrm: Jordan normal form for nilpotent}
  Es sei $V$ ein endlichdimensionaler $K$-Vektorraum und $f \colon V \to V$ ein nilpotenter Endomorphismus.
  \begin{enumerate}[leftmargin=*, label=\roman*)]
    \item
      Es gibt eine geordnete Basis $\basis{B}$ von $V$ und $n_1, \dotsc, n_s \geq 1$, so dass
      \[
        \Mat_\basis{B}(f)
        =
        \begin{pmatrix}
          J_{n_1} &         &         \\
                  & \ddots  &         \\
                  &         & J_{n_s}
        \end{pmatrix}.
      \]
    \item
      Die Zahlen $n_1, \dotsc, n_s$ sind eindeutig bis auf Permutation.
    \item
      Ist $f^N = 0$ für ein $N \geq 1$, so ist $n_i \leq N$ für alle $i = 1, \dotsc, s$.
  \end{enumerate}
\end{theorem}


\begin{corollary}
  Ist $A \in \Mat_n(K)$ nilpotent, so gibt es $S \in \GL_n(K)$ und $n_1, \dotsc, n_s \geq 1$ mit
  \[
    S A S^{-1}
    =
    \begin{pmatrix}
      J_{n_1} &         &         \\
              & \ddots  &         \\
              &         & J_{n_s}
    \end{pmatrix}.
  \]
  Dabei sind die Zahlen $n_1, \dotsc, n_s$ eindeutig bis auf Permutation, und ist $A^N = 0$ für ein $N \geq 1$, so ist $n_i \leq N$ für alle $i = 1, \dotsc, s$.
\end{corollary}


\begin{lemma}\label{lem: extensions to directs summands}
  Ist $V$ ein endlichdimensionaler $K$-Vektorraum, und sind $U, W, \subseteq V$ zwei Untervektorräume mit $U \cap W = 0$, so gibt es einen Untervektorraum $\overline{W} \subseteq V$ mit $W \subseteq \overline{W}$ und $V = U \oplus \overline{W}$.
\end{lemma}
\begin{proof}
  Es sei $\basis{B}_1 = (u_1, \dotsc, u_r)$ eine Basis von $U$ und $\basis{B}_2 = (w_1, \dotsc, w_s)$ eine Basis von $W$.
  Dann ist $\basis{B} \coloneqq (u_1, \dotsc, u_r, w_1, \dotsc, w_s)$ eine Basis von $U + W = U \oplus W$.
  Ergänze $\basis{B}$ zu einer Basis $\basis{C} = (u_1, \dotsc, u_r, w_1, \dotsc, w_s, w_{s+1}, \dotsc, w_t)$ von $V$ und setze $\overline{W} \coloneqq \bracket{w_1, \dotsc, w_t}$.
\end{proof}


\begin{proof}[Beweis der Existenz in Theorem~\ref{thrm: Jordan normal form for nilpotent}]
  Für alle $k \geq 0$ sei $N_k \coloneqq \ker(f^k)$.
  Es sei $p \coloneqq \min \{n \geq 0 \mid f^n\} = 0$ der Nilpotenzindex von $f$
  
  \begin{claim}
    Für alle $k \geq 0$ ist $N_{k+1} = f^{-1}(N_k)$, und somit insbesondere $f(N_{k+1}) \subseteq N_k$.
  \end{claim}
  \begin{proof}
    Es gilt
    \[
      v \in f^{-1}(N_k)
      \iff
      f(v) \in N_k
      \iff
      f^k(f(v)) = 0
      \iff
      f^{k+1}(v) = 0
      \iff
      v \in N_{k+1}.
    \]
  \end{proof}
  
  \begin{claim}
    Es Untervektorräume $W_1, \dotsc, W_p \subseteq V$ so dass
    \begin{enumerate}[leftmargin=*, label=\roman*)]
      \item
        $N_k = N_{k-1} \oplus W_k$ für alle $k = 1, \dotsc, p$,
      \item
        $f(W_k) \subseteq W_{k-1}$ für alle $k = 2, \dotsc, p$, und
      \item
        die Einschränkung $f|_{W_k}$ ist injektiv für alle $k = 2, \dotsc, p$.
    \end{enumerate}
  \end{claim}
  \begin{proof}
    Beginne mit $W_p \subseteq V$, so dass $V = N_p = N_{p-1} \oplus W_p$.
    Ist $W_{k+1}$ für ein $1 \leq k \leq p-1$ definiert, so ist
    \[
      f(W_{k+1}) \subseteq f(N_{k+1}) \subseteq N_k,
    \]
    und da
    \[
      f^{-1}(N_{k-1}) \cap W_{k+1}
      = N_k \cap W_{k+1}
      = 0,
    \]
    ist $f(W_{k+1}) \cap N_{k-1} = 0$.
    Nach Lemma~\ref{lem: extensions to directs summands} gibt es einen Untervektorraum $W_k \subseteq N_k$ mit $f(W_{k+1}) \subseteq W_k$ und $N_k = N_{k-1} \oplus W_k$.
    Die Injektivität von $f|_{W_k}$ für $k = 2, \dotsc, p$ folgt aus
    \[
        \ker(f|_{W_k})
      = \ker(f) \cap W_k
      = N_1 \cap W_k
      \subseteq N_{k-1} \cap W_k
      = 0.
      \qedhere
    \]
  \end{proof}
  
  Wähle eine Basis $\basis{B}_p = (v^p_1, \dotsc, v^p_{n_p})$ von $W_p$.
  Wegen der Injektivität von $f_{W_p}$ ist die Familie $f(\basis{B}_p) \coloneqq (f(v^p_1), \dotsc, f(v^p_{n_p}))$ linear unabhängig, und damit zu einer Basis
  \[
    \basis{B}_p = \left( f(v^p_1), \dotsc, f(v^p_{n_p}), v^{p-1}_1, \dotsc, v^{p-1}_{n_{p-1}} \right)
  \]
  von $W_{p-1}$ ergänzbar.
  Iteratives Fortführen liefert für $W_{p-i}$ eine Basis $\basis{B}_{p-i}$ de Form
  \[
    \renewcommand{\arraystretch}{1.5}
    \begin{matrix}
      f^i(v^p_1),     & \dotsc,  & f^i(v^p_{n_p})      \\
      f^{i-1}(v^p_1), & \dotsc,  & f^{i-1}(v^p_{n_p})  \\
      \vdots          &          & \vdots              \\
      v^{p-i}_1,      & \dotsc,  & v^{p-i}_{n_{p-i}}.
    \end{matrix}
  \]
  Da
  \begin{align*}
       V
     = N_p
     = N_{p-1} \oplus W_p
     = N_{p-2} \oplus W_{p-1} \oplus W_p
     = \dotsb
    &= N_0 \oplus W_1 \oplus \dotsb \oplus W_p \\
    &= W_1 \oplus \dotsb \oplus W_p
  \end{align*}
  ergibt sich durch Zusammenfügen der einzelnen Basen $\basis{B}_p, \dotsc, \basis{B}_1$ eine Basis $\tilde{\basis{B}}$ von $V$ von der Form
  \[
    \renewcommand{\arraystretch}{1.5}
    \setlength\arraycolsep{4pt}
    \begin{array}{cccccccccc}
      v^p_1,          & \dotsc, & v^p_{n_p},          &                     &         &                             &         &         &         &             \\
      f(v^p_1),       & \dotsc, & f(v^p_{n_p}),       & v^{p-1}_1,          & \dotsc, & v^{p-1}_{n_{p-1}},          &         &         &         &             \\
      \vdots          &         & \vdots              & \vdots              &         & \vdots                      &         &         &         &             \\
      f^{p-1}(v^p_1), & \dotsc, & f^{p-1}(v^p_{n_p}), & f^{p-2}(v^{p-1}_1), & \dotsc, & f^{p-2}(v^{p-1}_{n_{p-1}}), & \dotsc, & v^1_1,  & \dotsc, & v^1_{n_1}.
    \end{array}
  \]
  Trägt man nun dieses zweidimensionale Schema von oben nach unten, von links nach rechts in eine Familie $\basis{B}$ ein, so ist dies die gewünschte Matrix.
\end{proof}


[Hier das konkrete Vorgehen angeben.]










\subsection{Allgemeine Jordannormalform}


\begin{definition}
  Für einen Endomorphismus $f \colon V \to V$ und einen Skalar $\lambda \in K$ ist
  \[
    V^\sim_\lambda(f)
    \coloneqq
    \{
      v \in V
      \mid
      \text{es gibt $n \geq 1$ mit $(f - \lambda \id_V)^n(v) = 0$}
    \}
  \]
  der \emph{Hauptraum} von $f$ zu $\lambda$.
\end{definition}


\begin{lemma}
  Es sei $f \colon V \to V$ und $\lambda \in K$.
  \begin{enumerate}[leftmargin=*, label=\roman*)]
    \item
      Der Hauptraum $V^\sim_\lambda(f)$ ist ein Untervektorraum von $V$.
    \item
      Es gilt $V_\lambda(f) \subseteq V^\sim_\lambda(f)$.
    \item
      Es ist genau dann $V^\sim_\lambda(f) \neq 0$, wenn $\lambda$ ein Eigenwert von $f$ ist.
    \item
      Der Hauptraum $V^\sim_\lambda(f)$ ist $f$-invariant.
    \item
      Ist $V$ endlichdimensional, so gibt es $N \geq 1$ mit $(f - \lambda \id_V)^N(v) = 0$ für alle $v \in V^\sim_\lambda(f)$, und es gilt $V^\sim_\lambda(f) = \ker (f - \lambda \id_V)^N$.
  \end{enumerate}
\end{lemma}


\begin{lemma}
  Es sei $f \colon V \to V$ ein Endomorphismus eines endlichdimensionalen $K$-Vek\-tor\-raums $V$.
\end{lemma}


\begin{notation}
  Für alle $n \geq 1$ und $\lambda \in K$ ist
  \[
    J(n, \lambda)
    \coloneqq
    \begin{pmatrix}
      \lambda & 1       &         &         \\
              & \ddots  & \ddots  &         \\
              &         & \ddots  & 1       \\
              &         &         & \lambda
    \end{pmatrix}
    \in \Mat_n(K)
  \]
  der Jordanblock von Größe $n$ zu(m Eigenwert) $\lambda$.
\end{notation}


\begin{theorem}\label{thrm: general Jordan normal form}
  Es sei $f \colon V \to V$ ein Endomorphismus eines endlichdimensionalen $K$-Vektorraums $V$, so dass $V = V^\sim_{\lambda_1}(f) \oplus \dotsb \oplus V^\sim_{\lambda_t}(f)$ für die Eigenwerte $\lambda_1, \dotsc, \lambda_t \in K$ von $f$.
  \begin{enumerate}[leftmargin=*, label=\roman*)]
    \item
      Es gibt eine geordnete Basis $\basis{B}$ von $V$ und $n^{(1)}_1, \dotsc, n^{(1)}_{s_1}, \dotsc, n^{(t)}_1, \dotsc, n^{(t)}_{s_t} \geq 1$, so dass
      \[
        \Mat_\basis{B}(f)
        =
        \begin{pmatrix}
          J(n^{(1)}_1, \lambda_1) &         &                             &         &                         &         &                             \\
                                  & \ddots  &                             &         &                         &         &                             \\
                                  &         & J(n^{(1)}_{s_1}, \lambda_1) &         &                         &         &                             \\
                                  &         &                             & \ddots  &                         &         &                             \\
                                  &         &                             &         & J(n^{(t)}_1, \lambda_1) &         &                             \\
                                  &         &                             &         &                         & \ddots  &                             \\
                                  &         &                             &         &                         &         & J(n^{(t)}_{s_t}, \lambda_1)
        \end{pmatrix}
      \]
    \item
      Die Zahlen $(n^{(1)}_1, \dotsc, n^{(1)}_{s_1}), \dotsc, (n^{(t)}_1, \dotsc, n^{(t)}_{s_t})$ sind jeweils eindeutig bis auf Permutation.
    \item
      Es gilt $n^{(i)}_1 + \dotsb + n^{(i)}_{s_i} = \dim V^\sim_\lambda(f)$ für alle $i = 1, \dotsc, t$.
    \item
      Für alle $i = 1, \dotsc, t$ gilt $\max_{j=1,\dotsc,s_i} n^{(i)}_j = \min \{p \geq 0 \mid (f - \lambda_i \id_V)^p|_{V^\sim_{\lambda_i}(f)} = 0\}$.
  \end{enumerate}
\end{theorem}


[Hier das konkrete Vorgehen angeben]










\subsection{Existenz der Hauptraumzerlegung}


\begin{lemma}
  Es sei $f \colon V \to V$ ein Endomorphismus eines endlichdimensionalen $K$-Vek\-tor\-raums $V$.
  \begin{enumerate}[leftmargin=*, label=\roman*)]
    \item
      Für alle $\lambda, \mu \in K$ mit $\lambda \neq \mu$ ist die Einschränkung $(f - \lambda \id_V)|_{V^\sim_\mu(f)}$  invertierbar.
    \item
      Für alle $\lambda_1, \dotsc, \lambda_t \in K$ ist die Summe $V^\sim_{\lambda_1}(f) + \dotsb + V^\sim_{\lambda_t}(f)$ direkt.
  \end{enumerate}
\end{lemma}


\begin{lemma}
  Ist $f \colon V \to V$ ein Endomorphismus eines endlichdimensionalen $K$-Vek\-tor\-raums $V$ und sind $U, W \subseteq V$ zwei $f$-invariante Untervektorräume mit $V = U \oplus W$, so gilt
  \[
    \chi_f(T) = \chi_{f|_U}(T) \cdot \chi_{f|_W}(T).
  \]
\end{lemma}


\begin{lemma}\label{lem: finding direct summands to generalized eigenspaces}
  Es sei $f \colon V \to V$ ein Endomorphismus eines endlichdimensionalen Vektorraums $V$ und $\lambda \in K$.
  Ferner sei $U \subseteq V$ ein $f$-invarianter Untervektorraum mit $V = V^\sim_\lambda(f) \oplus U$.
  Dann ist $\lambda$ kein Eigenwert von $f|_U$, und es gilt
  \[
    \chi_f(T) = (T-\lambda)^{\dim V^\sim_\lambda(f)} \cdot \chi_{f|_U}(T).
  \]
\end{lemma}



\begin{lemma}[Fitting]
  Es sei $f \colon V \to V$ ein Endomorphismus eines endlichdimensionalen $K$-Vektorraums $V$.
  Für alle $k \geq 0$ sei
  \[
    N_k \coloneqq \ker f^k
    \quad\text{und}\quad
    R_k \coloneqq \im f^k.
  \]
  \begin{enumerate}[leftmargin=*, label=\roman*)]
    \item
      Es gilt
      \begin{gather*}
        0 = N_0 \subseteq N_1 \subseteq N_2 \subseteq N_3 \subseteq \dotsb
      \shortintertext{und}
        V = R_0 \supseteq R_1 \supseteq R_2 \supseteq R_3 \supseteq \dotsb
      \end{gather*}
    \item
      Für $k \geq 0$ sind die folgenden Bedingungen äquivalent:
      \begin{enumerate}[leftmargin=*, label=\alph*)]
        \item
          $N_{k+1} = N_k$,
        \item
          $N_l = N_k$ für alle $l \geq k$,
        \item
          $R_{k+1} = R_k$,
        \item
          $R_l = R_k$ für alle $l \geq k$.
      \end{enumerate}
      (Wenn also eine der beiden Ketten einmal stabiliert, so sind beide Ketten von dort an stabil.)
    \item
      Die beiden Teilmengen $N \coloneqq \bigcup_{k \geq 0} N_k$ und $R \coloneqq \bigcap_{k \geq 0} R_k$ sind $f$-invariante Untervektorräume von $V$, und es gilt $V = N \oplus R$.
  \end{enumerate}
\end{lemma}


\begin{theorem}[Existenz der Hauptraumzerlegung]
  Es sei $f \colon V \to V$ ein Endomorphismus eines endlichdimensionalen $K$-Vektorraums $V$.
  Dann gibt es genau dann eine Hauptraumzerlegung von $V$ bezüglich $f$, wenn das charakteristische Polynom $\chi_f(T)$ in Linearfaktoren zerfällt.
\end{theorem}


\begin{proof}
  Wenn es eine Hauptraumzerlegung gibt, dann ist $f$ nach Theorem~\ref{thrm: general Jordan normal form} trigonalisierbar.
  Somit zerfällt das charakteristische Polynom dann in Linearfaktoren.
  
  Die umgekehrte Aussage verläuft per Induktion über die Dimension von $V$.
  Für $\dim V = 0$ ist die Aussage klar.
  Für $\dim V > 0$ sei $\chi_f(T) = (T - \lambda_1)^{n_1} \dotsm (T - \lambda_s)^{n_s}$.
  Durch Anwenden von Fittings Lemma auf $f - \lambda_1 \id_V$ erhalten wir, dass
  \[
    V = V^\sim_{\lambda_1} \oplus R,
  \]
  wobei $R \subseteq V$ ein $f$-invarianter Untervektorraum ist.
  Nach Lemma~\ref{lem: finding direct summands to generalized eigenspaces} ist $\chi_{f|_R}(T) = (T - \lambda_2)^{n_2} \dotsm (T - \lambda_s)^{n_s}$, und nach Induktionsvoraussetzung ist
  \[
    R = R^\sim_{\lambda_2}(f|_R) \oplus \dotsb \oplus R^\sim_{\lambda_s}(f|_R),
  \]
  und somit insgesamt
  \[
              V
    =         V^\sim_{\lambda_1} \oplus R^\sim_{\lambda_2}(f|_R) \oplus \dotsb \oplus R^\sim_{\lambda_s}(f|_R)
    \subseteq V^\sim_{\lambda_1} \oplus V^\sim_{\lambda_2}(f|_R) \oplus \dotsb \oplus V^\sim_{\lambda_s}(f|_R)
    \subseteq V.
  \]
\end{proof}


\begin{corollary}
  Ist $K$ ein algebraisch abgeschlossener Körper und $f \colon V \to V$ ein Endomorphismus eines endlichdimensionalen $K$-Vektorraums $V$, so ist $V = \bigoplus_{\lambda \in K} V^\sim_\lambda(f)$.
\end{corollary}


\begin{corollary}[Cayley-Hamilton für algebraisch abgeschlossene Körper]
  Ist $K$ ein algebraisch abgeschlossener Körper und $f \colon V \to V$ ein Endomorphismus eines endlichdimensionalen $K$-Vektorraums $V$ (bzw.\ $A \in \Mat_n(K)$), so ist $\chi_f(f) = 0$ (bzw.\ $\chi_A(A) = 0$).
\end{corollary}


\begin{remark}
  Für jeden Körper $K$ gibt es einen algebraisch abgeschlossenen Körper $\overline{K}$, so dass $K$ ein Unterkörper von $\overline{K}$ ist.
  Der Satz von Cayley-Hamilton gilt daher für beliebige Körper.
  Er kann auch ohne die Jordan-Normalform bewiesen werden (siehe etwa Fischer).
\end{remark}


\begin{corollary}[Abstrakte Jordanzerlegung]
  Ist $K$ ein algebraisch abgeschlossener Körper und $f \colon V \to V$ ein Endomorphismus eines endlichdimensionalen $K$-Vektorraums $V$ (bzw.\ $A \in \Mat_n(K)$), so gibt es eindeutige Endomorphismen $d, n \colon V \to V$ (bzw.\ $D, N \in \Mat_n(K)$), so dass
  \begin{enumerate}[leftmargin=*, label=\alph*)]
    \item
      $f = d + n$ (bzw.\ $A = D + N$),
    \item
      $d$ (bzw.\ $D$) ist diagonalisierbar und $n$ (bzw.\ $N$) ist nilpotent,
    \item
      $d$ und $n$ (bzw.\ $D$ und $N$) kommutieren.
  \end{enumerate}
\end{corollary}


[Falls noch Zeit: Idee der Hauptraumzerlegung genauer erläutern, und allgemeines Lemma nennen.]










\subsection{Homogene, lineare, gewöhnliche Differentialgleichungen}


Ist $A \in \Mat_n(\Complex)$, so lassen sich die komplexen Lösungen der Differentialgleichung
\[
  A y = y'
\]
für $y \colon \Reals \to \Complex^n$ wie folgt finden:

Der Lösungsraum wird von den Spalten der Matrix $\exp(At)$ aufgefasst (wenn man diese als Funktionen von $t$ betrachtet).
Ist $S^{-1} A S = J$ eine Jordannormalform von $A$ und $J = D + N$ die entsprechende Jordanzerlegung, so ist
\[
    \exp(At)
  = \exp(S Jt S^{-1})
  = S \exp(Jt) S^{-1}
  = S \exp(Dt) \exp(Nt) S^{-1},
\]
wobei im letzen Schritt genutzt wird, dass $D$ und $N$ kommutieren.










