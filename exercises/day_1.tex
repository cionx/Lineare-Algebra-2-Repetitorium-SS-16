\documentclass[a4paper, 10pt]{scrartcl}

\usepackage{microtype}
\usepackage{../generalstyle}
\usepackage{exercisestyle}


\title{Lineare Algebra II Repetitorium \\ Übungen, Tag 1}
\author{Jendrik Stelzner}
\date{\today}


\begin{document}
\maketitle


\begin{question}
  Es sei $V$ ein $K$-Vektorraum und $\id \coloneqq \id_V$.
  \begin{enumerate}[leftmargin=*]
    \item
      Es sei $n \colon V \to V$ ein nilpotenter Endomorphismus.
      Zeigen Sie, dass $\id - n$ invertierbar ist.
      (\emph{Hinweis}: Betrachten Sie Endomorphismen der Form $\id + n + n^2 + \dotsb + n^k$.)
    \item
      Zeigen Sie ferner, dass $\lambda \id - n$ für alle $\lambda \neq 0$ invertierbar ist.
    \item
      Es sei nun $f \colon V \to V$ ein beliebiger Endomorphismus und $V$ endlichdimensional.
      Zeigen Sie für $\lambda, \mu \in K$ mit $\lambda \neq \mu$, dass die Einschränkung $(f - \lambda \id)|_{V^\sim_\mu(f)}$ invertierbar ist.
  \end{enumerate}
\end{question}


\begin{question}
  Bestimmen Sie für die folgenden Matrizen jeweils eine Jordannormalform über dem angegebenen Körper, inklusive einer entsprechenden Jordanbasis.
  \begin{enumerate}
    \item
      $
      A
      \coloneqq
      \begin{pmatrix*}[r]
        0 & 0 &  1  \\
        1 & 0 & -3  \\
        0 & 1 &  3
      \end{pmatrix*}
      $
      für $K = \Reals$.
    \item
      $
      B
      \coloneqq
      \begin{pmatrix}
        0 & 0 & 1 \\
        1 & 1 & 1 \\
        1 & 0 & 0
      \end{pmatrix}
      $
      und $K = \Integers/2$.
    \item
      $
      C
      \coloneqq
      \begin{pmatrix*}[r]
        1 & 0 & 1 & -1  \\
        0 & 1 & 1 &  0  \\
        0 & 0 & 1 &  1  \\
        0 & 0 & 0 &  1
      \end{pmatrix*}
      $
      für $K = \Reals$.
    \item
      $
      D
      \coloneqq
      \begin{pmatrix*}[r]
        -6 & 12 & 4 \\
        -2 &  2 & 1 \\
        -8 & 20 & 6
      \end{pmatrix*}
      $
      für $K = \Reals$.
    \item
      $
      E
      \coloneqq
      \begin{pmatrix}
        0 & 0 & 1 & 2 \\
        0 & 1 & 2 & 1 \\
        2 & 1 & 0 & 0 \\
        0 & 2 & 2 & 0
      \end{pmatrix}
      $
      für $K = \Integers/3$.
  \end{enumerate}
\end{question}


\begin{solution}
  \begin{enumerate}
    \item
      Das charakteristische Polynom:
      \[
        T^3 - 3 T^2 + 3T - 1 = (T - 1)^3
      \]
      Eine mögliche Jordanbasis:
      \[
        \left( \cvector{1 \\ -2 \\ 1}, \cvector{-1 \\ 1 \\ 0}, \cvector{1 \\ 0 \\ 0} \right)
      \]
      Entsprechend Jordannormalform:
      \[
        \begin{pmatrix}
          1 & 1 &   \\
            & 1 & 1 \\
            &   & 1
        \end{pmatrix}
      \]
    \item
      Das charakteristische Polynom:
      \[
        T^3 + T^2 + T + 1 = (T - 1)^3
      \]
      Eine mögliche Jordanbasis:
      \[
        \left( \cvector{1 \\ 1  \\ 1}, \cvector{1 \\ 0 \\ 0}, \cvector{1 \\ 0 \\ 1} \right)
      \]
      Entsprechend Jordannormalform:
      \[
        \begin{pmatrix}
          1 & 1 &   \\
            & 1 &   \\
            &   & 1
        \end{pmatrix}
      \]
    \item
      Das charakteristische Polynom:
      \[
        (T - 1)^4
      \]
      Eine mögliche Jordanbasis:
      \[
        \left( \cvector{1 \\ 0 \\ 0 \\ 0}, \cvector{1 \\ 1 \\ 0 \\ 0}, \cvector{-1 \\ 0 \\ 1 \\ 0}, \cvector{0 \\ 0 \\ 0 \\ 1} \right)
      \]
      Entsprechend Jordannormalform:
      \[
        \begin{pmatrix}
          1 &   &   &   \\
            & 1 & 1 &   \\
            &   & 1 & 1 \\
            &   &   & 1
        \end{pmatrix}
      \]
    \item
      Das charakteristische Polynom:
      \[
        T^3 - 2 T^2 = T^2 (T - 2)
      \]
      Eine mögliche Jordanbasis:
      \[
        \left( \cvector{12 \\ -6 \\ 36}, \cvector{8 \\ 5 \\ 0}, \cvector{1 \\ 0 \\ 2} \right)
      \]
      Entsprechend Jordannormalform:
      \[
        \begin{pmatrix}
          0 & 1 &   \\
            & 0 &   \\
            &   & 2
        \end{pmatrix}
      \]
    \item
      Das charakteristische Polynom:
      \[
        T^4 + 2 T^3 + T + 2 = (T - 2)^3 (T - 1)
      \]
      Eine mögliche Jordanbasis:
      \[
        \left(
          \cvector{0 \\ 1 \\ 1 \\ 1},
          \cvector{2 \\ 2 \\ 0 \\ 2},
          \cvector{0 \\ 2 \\ 1 \\ 2},
          \cvector{0 \\ 1 \\ 0 \\ 0}
        \right)
      \]
      Die entsprechende Jordannormalform:
      \[
        \begin{pmatrix}
          1 &   &   &   \\
            & 2 & 1 &   \\
            &   & 2 & 1 \\
            &   &   & 2
        \end{pmatrix}
      \]
  \end{enumerate}
\end{solution}


\begin{question}
  Es sei $V$ ein endlichdimensionaler $K$-Vektorraum.
  \begin{enumerate}[leftmargin=*]
    \item
      Es seien $f, g \colon V \to V$ zwei Endomorphismen.
      Zeigen Sie, dass
      \[
        \dim \ker(f \circ g) \leq \dim \ker(f) + \dim \ker(g).
      \]
      (\emph{Hinweis}: Zeigen Sie zunächst, dass $g(\ker(f \circ g)) \subseteq \ker(f)$.)
    \item
      Folgern Sie, dass für jeden Endomorphismus $f \colon V \to V$ die folgende Ungleichung gilt:
      \[
        \dim \ker(f^n) \leq n \cdot \dim \ker(f).
      \]
    \item
      Entscheiden Sie, ob die folgende Aussage wahr oder falsch ist:
      
      Wenn $f^{10} = 0$ und $\dim \ker(f) < 10$, dann ist $\dim V \leq 2016$.
  \end{enumerate}
\end{question}


\begin{question}
  Es sei $e \colon V \to V$ ein Endomorphismus eines $K$-Vektorraums $V$ mit $e^2 = e$.
  Zeigen Sie, dass $V = \im(e) \oplus \ker(e)$.
\end{question}


\begin{question}
  Es sei $f \colon V \to V$ ein Endomorphismus eines endlichdimensionalen $K$-Vektorraums $V$.
  Für alle $k \geq 0$ sei $N_k \coloneqq \ker(f^k)$ und $R_k \coloneqq \im(f^k)$.
  \begin{enumerate}[leftmargin=*]
    \item
      Zeigen Sie, dass $N_k \subseteq N_{k+1}$ und $R_k \supseteq R_{k+1}$ für alle $k \geq 0$, dass also
      \begin{gather*}
        0 = N_0 \subseteq N_1 \subseteq N_2 \subseteq \dotso
      \quad\text{und}\quad
        V = R_0 \supseteq R_1 \supseteq R_2 \supseteq \dotso
      \end{gather*}
    \item
      Zeigen Sie, dass genau dann $N_{k+1} = N_k$, wenn $R_{k+1} = R_k$.
    \item
      Es sei $k \geq 1$ mit $R_{k+1} = R_k$.
      Zeigen Sie, dass $R_{k+i} = R_k$ für alle $i \geq 0$.
      Folgern Sie, dass auch $N_{k+i} = N_k$ für alle $i \geq 0$.
    \item
      Es seien $N \coloneqq \bigcup_{k \geq 0} N_k$ und $R \coloneqq \bigcap_{k \geq 0} R_k$.
      \begin{enumerate}
        \item
          Zeigen Sie, dass $N$ und $R$ invariant unter $f$ sind.
        \item
          Zeigen Sie, dass $V = N \oplus R$.
      \end{enumerate}
      (\emph{Hinweis}: Zeigen Sie zunächst, dass es ein $k_0 \geq 0$ gibt, so dass $R = R_{k_0}$ und $N = N_{k_0}$).
  \end{enumerate}
\end{question}


\begin{question}
  Bestimmen Sie die Lösungsräume der folgenden homogenen linearen Differentialgleichungen.
  Dabei seien $f, g, h \in C^\infty(\Reals)$.
  \[
    \left\{
      \begin{array}{ccrr}
        f'  & = & -f  & - 6g, \\
        g'  & = & 2f  & + 6g;
      \end{array}
    \right.
    \quad
    \left\{
      \begin{array}{ccrr}
        f'  & = & -f  & - g,  \\
        g'  & = & 2f  & + g;
      \end{array}
    \right.
    \quad
    \left\{
      \begin{array}{ccrrr}
        f'  & = & 2f  & + 2g  & + 3h, \\
        g'  & = &  f  & + 3g  & + 3h, \\
        h'  & = & -f  & - 2f  & - 2h.
      \end{array}
    \right.
  \]
\end{question}


\begin{solution}
  \begin{enumerate}
    \item
      Charakteristisches Polynom:
      \[
        T^2 - 5T + 6 = (T-2)(T-3)
      \]
      Jordanbasis:
      \[
        \left( \cvector{-2 \\ 1}, \cvector{-3 \\ 2} \right).
      \]
      Entsprechende Jordannormalform:
      \[
        \begin{pmatrix}
          2 &   \\
            & 3
        \end{pmatrix}
      \]
      Die Basiswechselmatrix ist selbstinvers.
      Matrixexponential:
      \[
        \begin{pmatrix}
           4 e^{2t} - 3 e^{3t}  &  6 e^{2t} - 6 e^{3t}  \\
          -2 e^{2t} + 2 e^{3t}  & -3 e^{2t} + 4 e^{3t}
        \end{pmatrix}
      \]
    \item
      Charakteristisches Polynom:
      \[
        T^2 + 1 = (T-i)(T+i)
      \]
      Jordanbasis:
      \[
        \left( \cvector{-1+i \\ 2}, \cvector{-1-i \\ 2} \right).
      \]
      Entsprechende Jordannormalform:
      \[
        \begin{pmatrix}
          -i  &   \\
              & i 
        \end{pmatrix}
      \]
      Inverse der Basiswechselmatrix:
      \[
        \frac{1}{4}
        \begin{pmatrix*}[r]
          -2i & 1-i \\
           2i & 1+i
        \end{pmatrix*}
      \]

      Matrixexponential:
      \[
        \begin{pmatrix}
          \cos(t) - \sin(t) & -\sin(t)            \\
          2 \sin(t)         &  \cos(t) + \sin(t)
        \end{pmatrix}
      \]
    \item
      Charakteristisches Polynom:
      \[
        T^3 - 3 T^2 + 3 T - 1 = (T - 1)^3
      \]
      Jordanbasis:
      \[
        \left( \cvector{0 \\ 1 \\ -\frac{2}{3}}, \cvector{3 \\ 3 \\ 3}, \cvector{0 \\ 0 \\ 1} \right)
      \]
      Entsprechende Jordannormalform:
      \[
        \begin{pmatrix}
          1 &   &   \\
            & 1 & 0 \\
            &   & 1
        \end{pmatrix}
      \]
      Inverse der Basiswechselmatrix:
      \[
        \frac{1}{3}
        \begin{pmatrix}
          -3  & 3 & 0 \\
           1  & 0 & 0 \\
           1  & 2 & 3
        \end{pmatrix}
      \]
      Matrixexponential:
      \[
        e^t
        \begin{pmatrix}
          1 + t &     2t  &     3t  \\
              t & 1 + 2t  &     3t  \\
             -t &    -2t  & 1 - 3t
        \end{pmatrix}
      \]
  \end{enumerate}
\end{solution}


% \begin{question}
%   \begin{enumerate}[leftmargin=*]
%     \item
%       Geben Sie ein Beispiel für einen $\Reals$-Vektorraum $V$, einen Endomorphismus $f \colon V \to V$ und ein $\lambda \in \Reals$, so dass die Einschränkung $(f - \lambda \id)|_{V^\sim_\lambda(f)}$ nicht nilpotent ist.
%     \item
%       Geben Sie ein Beispiel für einen $\Reals$-Vektorraum $V$, einen Endomorphismus $f \colon V \to V$ und einen $f$-invarianten Untervektorräum $U \subseteq V$, so dass es keinen $f$-invarianten Untervektorraum $W \subseteq V$ mit $V = U \oplus W$ gibt.
%   \end{enumerate}
% \end{question}


% \begin{question}
%   Entscheiden Sie, ob die folgenden Aussagen wahr oder falsch sind:
%   \begin{enumerate}[leftmargin=*]
%     \item
%       Ist $f \colon V \to V$ ein nilpotenter Endomorphismus eines Vektorraums $V$, so hat ist $0$ der eindeutige Eigenwert von $f$.
%     \item
%       Ist $f \colon V \to V$ ein nilpotenter Endomorphismus eines Vektorraums $V$
%       Wenn $\im(f)$ endlichdimensional ist, so ist $V$ endlichdimensional.
%     \item
%       
%   \end{enumerate}
% \end{question}


\newpage


\printsolutions



\end{document}
