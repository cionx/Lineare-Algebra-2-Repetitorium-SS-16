\documentclass[a4paper, 10pt]{scrartcl}

\usepackage{../generalstyle}
\usepackage{exercisestyle}


\title{Lineare Algebra II Repetitorium \\ Übungen, Tag 2}
\author{Jendrik Stelzner}
\date{\today}


\begin{document}
\maketitle


\begin{question}
  Es seien $f, g \colon V \to V$ zwei kommutierende Endomorphismen eines $K$-Vektorraums $V$.
  \begin{enumerate}[leftmargin=*]
    \item
      Zeigen Sie, dass $V_\lambda(f)$ für alle $\lambda \in K$ invariant unter $g$ ist.
    \item
      Entscheiden Sie, ob auch $V^\sim_\lambda(f)$ für alle $\lambda \in K$ invariant unter $g$ ist.
  \end{enumerate}
  Es seien nun $H, E \colon V \to V$ zwei Endomorphismen mit $HE - EH = 2E$.
  \begin{enumerate}[leftmargin=*, resume]
    \item
      Zeigen Sie, dass $E(V_\lambda(H)) \subseteq V_{\lambda + 2}(H)$ für alle $\lambda \in K$.
  \end{enumerate}
\end{question}


% \begin{question}
%   Es seien $f_1, \dotsc, f_n \colon V \to V$ Endomorphismen eines endlichdimensionalen $K$-Vektorraums $V$.
%   \begin{enumerate}[leftmargin=*]
%     \item
%       Es sei
%       $
%         V = \bigoplus_{\lambda_1, \dotsc, \lambda_n \in K}
%             (V_{\lambda_1}(f) \cap \dotsb \cap V_{\lambda_n}(f))
%       $.
%       Zeigen Sie, dass $V$ eine Basis aus gemeinsamen Eigenvektoren von $f_1, \dotsc, f_n$ besitzt.
%     \item
%       Umgekehrt besitze nun $V$ eine Basis aus gemeinsamen Eigenvektoren von $f_1, \dotsc, f_n$.
%       Zeigen Sie, dass
%       $
%         V = \bigoplus_{\lambda_1, \dotsc, \lambda_n \in K}
%             (V_{\lambda_1}(f) \cap \dotsb \cap V_{\lambda_n}(f)).
%       $
%   \end{enumerate}
% \end{question}


\begin{question}
  Beweisen Sie die Chauchy-Schwarz-Ungleichung.
\end{question}


\begin{solution}
  Es seien $v, w \in V$.
  Die Fälle $v = 0$ und $w = 0$ sind klar, es genügt daher den Fall $v, w \neq 0$ zu betrachten.
  Es ist
  \[
          0
    \leq  \bracket{ v - \frac{\bracket{v,w}}{\|w\|^2} w, v - \frac{\bracket{v,w}}{\|w\|^2} w}
    =     \|v\|^2 - \frac{|\bracketnoscale{v,w}|^2}{\|w\|^2}
  \]
  und somit $|\bracketnoscale{v,w}| \leq \|v\| \|w\|$.
  
  Sind $v$ und $w$ linear abhängig, so gilt $w = \lambda v$ für ein $\lambda \in \Korper$, und durch Einsetzen ergibt sich $|\bracketnoscale{v,w}| = \|v\| \|w\|$.
  Gilt andererseits die Gleichheit $\bracket{v,w} = \|v\| \|w\|$, so gilt in der obigen Zeile $0 = v - \bracketnoscale{v,w}/\|w\|^2$, weshalb $v$ und $w$ linear abhängig sind.
\end{solution}


\begin{question}
  Es sei $V$ ein Skalarproduktraum und $(v_i)_{i \in I}$ eine Familie von Vektoren mit $v_i \neq 0$ für alle $i \in I$.
  Zeigen Sie, dass die Familie $(v_i)_{i \in I}$ linear unabhängig ist, wenn sie orthogonal ist.
  Entscheiden Sie, ob auch die Umkehrung gilt.
\end{question}


\begin{question}
  Es sei $V$ eine endlichdimensionaler Skalarproduktraum.
  \begin{enumerate}[leftmargin=*]
    \item
      Zeigen Sie, dass eine linear unabhängige Famlie $(v_1, \dotsc, v_m)$ genau dann orthonormal ist, wenn sie sich durch das Anwenden des Gram-Schmidt-Verfahrens nicht ändert.
    \item
      Zeigen Sie, dass sich jede orthonormale Familie $(v_1, \dotsc, v_m)$ von Vektoren $v_i \in V$ zu einer Orthonormalbasis $(v_1, \dotsc, v_m, v_{m+1}, \dotsc, v_n)$ von $V$ ergänzen lässt.
    \item
      Folgern Sie, dass $V$ eine Orthonormalbasis besitzt.
    \item
      Zeigen Sie, dass für jeden Untervektorraum $U \subseteq V$ die Gleichheit $V = U \oplus U^\perp$ gilt.
  \end{enumerate}
\end{question}


\begin{question}
  Es sei $V$ ein Skalarproduktraum.
  \begin{enumerate}[leftmargin=*]
    \item
      Zeigen Sie, dass die Abbildung
      \[
        \Phi \colon V \to V^*,
        \quad
        v \mapsto \bracket{-, v}
      \]
      wohldefiniert und $\Reals$-linear, bzw.\ $\Complex$-antilinear ist.
    \item
      Zeigen Sie, dass $\Phi$ injektiv ist.
  \end{enumerate}
  Von nun an sei $V$ zusätzlich endlichdimensional, und $\basis{B} = (v_1, \dotsc, v_n)$ sei eine Orthonormalbasis von $V$.
  \begin{enumerate}[leftmargin=*, resume]
    \item
      Folgern Sie, dass $\Phi$ ein Isomorphismus (für $\Korper = \Reals$), bzw.\ ein Antiisomorphismus ist.
    \item
      Zeigen Sie, dass $\Phi(\basis{B}) \coloneqq (\Phi(v_1), \dotsc, \Phi(v_n))$ mit der zu $\basis{B}$ dualen Basis $\basis{B}^* = (v_1^*, \dotsc, v_n^*)$ von $V^*$ übereinstimmt.
  \end{enumerate}
\end{question}


% \begin{question}
%   Es sei $V$ ein endlichdimensionaler Skalarproduktraum und $(v_1, \dotsc, v_n)$ eine Orthonormalbasis von $V$.
%   \begin{enumerate}[leftmargin = *]
%     \item
%       Zeigen Sie, dass $w = \sum_{i=1}^n \bracket{w, v_i} v_i$ für alle $w \in W$.
%     \item
%       Zeigen Sie, dass $\bracket{w_1, w_2} = \sum_{i=1}^n \bracket{w_1, v_i} \bracket{v_i, w_2}$ für alle $w_1, w_2 \in V$.
%     \item
%       Zeigen Sie, dass $\|w\|^2 = \sum_{i=1}^n |\bracketnoscale{w,v_i}|^2$ für alle $w \in W$.
%   \end{enumerate}
% \end{question}


\begin{question}
  Es seien $V$ und $W$ zwei endichdimensionale Skalarprodukträume und $\Phi_V \colon V \to V^*$ und $\Phi_W \colon W \to W^*$ die zugehörigen Isomorphismen mit $\Phi_V(v) = \bracket{-,v}$ und $\Phi_W(w) = \bracket{-,w}$ für alle $v \in V$ und $w \in W$.
  
  Es sei $f \colon V \to W$ eine lineare Abbildung, und $f^T \colon W^* \to V^*$ die zugehörige duale Abbildung, d.h.\ es ist $f^T(\psi) \coloneqq \psi \circ f$ für alle $\psi \in W^*$.
  
  Zeigen Sie, dass die adjungierte Abbildung $f^* \colon W \to V$ die eindeutig Abbildung ist, die das folgende Diagramm zum kommutieren bringt:
  \[
    \begin{tikzcd}[row sep = large, column sep = large, ampersand replacement=\&]
          W
          \arrow{r}{f^*}
          \arrow{d}[left]{\Phi_W}
      \&  V
          \arrow{d}{\Phi_V}
      \\
          W^*
          \arrow{r}{f^T}
      \&  V^*
    \end{tikzcd}
  \]
\end{question}


\begin{question}
  Es seien $V$ und $W$ zwei endlichdimensionale Skalarprodukträume.
  Es sei $\basis{B}$ eine geordnete Orthonormalbasis von $V$ und $\basis{C}$ eine geordnete Orthonormalbasis von $W$.
  Zeigen Sie, dass für jede $\Korper$-lineare Abbildung $f \colon V \to W$ die Gleichheit $\Mat_{\basis{C}, \basis{B}}(f^*) = \Mat_{\basis{B}, \basis{C}}(f)^*$ gilt.
\end{question}



\newpage


\printsolutions



\end{document}
