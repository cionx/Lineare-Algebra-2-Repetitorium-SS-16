\documentclass[a4paper, 10pt]{scrartcl}

\usepackage{../generalstyle}
\usepackage{exercisestyle}


\title{Lineare Algebra II Repetitorium \\ Übungen, Tag 3}
\author{Jendrik Stelzner}
\date{\today}


\begin{document}
\maketitle


Im Folgenden seien alle auftretenden Skalarprodukträume endlichdimensional.


\begin{question}
  Es sei $f \colon V \to V$ ein normaler Endomorphismus eines Skalarproduktraums $V$.
  Zeigen Sie:
  \begin{enumerate}[leftmargin=*]
    \item
      Für alle $v \in V$ ist $\|f(v)\| = \|f^*(v)\|$.
    \item
      Es ist $V_\lambda(f^*) = V_{\overline{\lambda}}(f)$ für alle $\lambda \in \Korper$.
    \item
      Für alle $\lambda, \mu \in \Korper$ mit $\lambda \neq \mu$ ist $V_\lambda(f) \perp V_\mu(f)$.
    \item
      Für $v \in V$ und $n \geq 1$ mit $f^n(v) = 0$ ist bereits $f(v) = 0$.
    \item
      Folgern Sie, dass $V^\sim_\lambda(f) = V_\lambda(f)$ für alle $\lambda \in K$.
  \end{enumerate}
  Zeigen Sie außerdem:
  \begin{enumerate}[leftmargin=*, resume]
    \item
      Ein Untervektorraum $U \subseteq V$ ist genau dann $f$-invariant, wenn $U^\perp$ invariant unter $f^*$ ist.
    \item
      Es ist $\im f^* = (\ker f)^\perp$ und $\ker f^* = (\im f)^\perp$.
  \end{enumerate}
\end{question}


\begin{question}
  Zeigen Sie, dass für einen Endomorphismus $f \colon V \to V$ eines Skalarproduktraums $V$ die folgenden Bedingungen äquivalent sind:
  \begin{enumerate}[leftmargin=*]
    \item
      Es gilt $f f^* = \id_V$.
    \item
      Es gilt $f^* f = \id_V$.
    \item
      $f$ ist ein Isomorphismus mit $f^* = f^{-1}$.
    \item
      Für alle $v_1, v_2 \in V$ ist $\bracket{f(v_1), f(v_2)} = \bracket{v_1, v_2}$.
    \item
      Für alle $v \in V$ ist $\|f(v)\| = \|v\|$.
  \end{enumerate}
\end{question}


\begin{question}
  Es sei $A \in \Mat_n(\Korper)$ .
  \begin{enumerate}[leftmargin=*]
    \item
      Zeigen Sie, dass genau dann $A^* A = I$, wenn die Spalten von $A$ eine Orthonormalbasis von $\Korper^n$ bilden.
    \item
      Zeigen Sie, dass genau dann $A A^* = I$, wenn die Zeilen von $A$ eine Orthonormalbasis von $\Korper^n$ bilden.
  \end{enumerate}
\end{question}


\begin{question}
  Es sei $f \colon V \to V$ ein Endomorphismus eines Skalarproduktraums $V$.
  Zeigen Sie die folgende Aussagen ohne Verwendung entsprechender Normalenformen:
  \begin{enumerate}[leftmargin=*]
    \item
      Ist $f$ selbstadjungiert, so sind alle Eigenwerte von $f$ reell.
    \item
      Ist $f$ antiselbstadjungiert, so sind alle Eigenwerte von $f$ rein imaginär.
    \item
      Ist $f$ unitär, so haben alle Eigenwerte von $f$ Betrag $1$.
  \end{enumerate}
\end{question}


\begin{question}
  Es sei $n \geq 1$.
  \begin{enumerate}[leftmargin=*]
    \item
      Zeigen Sie, dass $\det(U) \subseteq S^1$ für alle $U \in \Unitary(n)$, und dass $\det \colon \Unitary(n) \to S^1$ surjektiv ist.
    \item
      Zeigen Sie, dass $\det(O) \subseteq \{1,-1\}$ für alle $O \in \Orthogonal(n)$, und dass $\det \colon \Orthogonal(n) \to \{1,-1\}$ surjektiv ist.
  \end{enumerate}
\end{question}


\begin{question}
  Es sei $V$ ein Skalarproduktraum.
  \begin{enumerate}[leftmargin=*]
    \item
      Es sei $v \in V$ ein normierter Vektor.
      Zeigen Sie, dass die Abbildung $P_v \colon V \to V$ mit $P_v(w) = \bracket{w,v} v$ die orthogonale Projektion auf die Gerade $\Ell(v)$ ist.
    \item
      Es sei $(v_1, \dotsc, v_n)$ eine orthonormale Familie von Vektoren $v_1, \dotsc, v_n \in V$.
      Zeigen Sie, dass $P \coloneqq P_{v_1} + \dotsb + P_{v_n}$ die orthogonale Projektion auf den Untervektorraum $\Ell(v_1, \dotsc, v_n)$ ist.
  \end{enumerate}
\end{question}



\begin{question}
  Es sei $f \colon V \to V$ ein Endomorphismus eines Skalarproduktraums $V$.
  \begin{enumerate}[leftmargin=*]
    \item
      Zeigen Sie, dass $f$ genau dann positiv ist, wenn $\bracket{f(v), v} > 0$ für alle $v \in V$.
    \item
      Zeigen Sie, dass $f f^*$ und $f^* f$ positiv selbstadjungiert sind.
  \end{enumerate}
\end{question}


% \begin{question}
%   \begin{enumerate}[leftmargin=*]
%     \item
%       Es sei $V$ ein $K$-Vektorraum, wobei $\ringchar(K) \neq 2$ gilt.
%       Zeigen Sie, dass für jede symmetrische Bilinearform $\bracket{\cdot, \cdot} \colon V \times V \to K$ die Polarisationsformel
%       \[
%           \bracket{v_1, v_2}
%         = \frac{q(v_1 + v_2) - q(v_1) - q(v_2)}{2}
%         \quad
%         \text{für alle $v_1, v_2 \in V$}
%       \]
%       gilt, wobei $q \colon V \to K$, $v \mapsto \bracket{v,v}$ die zu $\bracket{\cdot, \cdot}$ gehörige qudratische Form ist.
%     \item
%       Es sei $V$ ein $\Complex$-Vektorraum und $\bracket{\cdot, \cdot} \colon V \times V \to K$ eine hermitsche Sesquilinearform und $q \colon V \to \Reals$, $v \mapsto \bracket{v,v}$.
%       Leiten Sie eine entsprechende Polarisationsformel her
%   \end{enumerate}
% \end{question}


Übungszettel finden sich online unter folgender URL:\\
\centerline{\url{https://github.com/cionx/Lineare-Algebra-2-Repetitorium-SS-16}}


\newpage


\printsolutions



\end{document}
