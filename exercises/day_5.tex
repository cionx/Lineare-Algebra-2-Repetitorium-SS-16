\documentclass[a4paper, 10pt]{scrartcl}

\usepackage{../generalstyle}
\usepackage{exercisestyle}


\title{Lineare Algebra II Repetitorium \\ Übungen, Tag 4}
\author{Jendrik Stelzner}
\date{\today}


\begin{document}
\maketitle


\begin{question}
  Skizzieren Sie $S^1$, $\Unitary(n)$ und $\SOrthogonal(2)$.
\end{question}



\begin{question}
  Entscheiden Sie, welche der folgenden Aussagen gelten für alle $n \geq 1$ gelten.
  \begin{enumerate}[leftmargin=*]
    \item
      Die Wegzusammenhangskomponenten von $\GL_n(\Reals)$ sind die beiden Untergruppen
      \[
        \GL_n(\Reals)_+ = \{ S \in \GL_n(\Reals) \mid \det S > 0 \}
        \quad\text{und}\quad
        \GL_n(\Reals)_- = \{ S \in \GL_n(\Reals) \mid \det S < 0 \}.
      \]
    \item
      Für alle $A, B \in \GL_n(\Reals)$ liegen entweder $A$ und $B$ in derselben Zusammenhangskomponente, oder $A$ und $-B$ liegen in der gleichen Zusammenhangskomponente.
    \item
      Von den beiden Wegzusammenhangskomponente von $\GL_n(\Reals)$ ist $\Orthogonal(n)$ diejenige, die die Einheitsmatrix enthält.
    \item
      Die schiefsymmetrischen Matrizen $\mathfrak{o}_n(\Reals) = \{ A \in \Mat_n(\Reals) \mid A^T = -A \}$ bilden eine wegzusammenhängende Teilmenge von $\Mat_n(\Rbb)$.
    \item
      Ist $n \geq 2$, so hat die Gruppe $\Unitary(n) \cap \GL_n(\Rbb)_+$ genau zwei Wegzusammenhangskomponenten.
    \item
      Es ist $G = \{ S \in \GL_n(\Rbb) \mid S^{-1} = -S \}$ eine zusammenhängende, aber nicht wegzusammenhängende Untergruppe von $\GL_n(\Rbb)$.
    \item
      Jede Untergruppe von $\GL_n(\Cbb)$ ist wegzusammenhängend.
    \item
      Die Menge der Drehmatrizen
      \[
        D
        \coloneqq
        \left\{
          \begin{pmatrix*}[r]
            \cos \varphi  & -\sin \varphi \\
            \sin \varphi  &  \cos \varphi 
          \end{pmatrix*}
        \,\middle|\,
        \varphi \in \Rbb
        \right\}
      \]
      ist eine wegzusammenhängende Untergruppe von $\GL_2(\Rbb)$.
  \end{enumerate}
\end{question}


\begin{question}
  Es sei $V$ ein $K$-Vektorraum und $[-,-] \colon V \times V \to V$ eine alternierend bilineare Abbildung.
  Für jedes $x \in V$ sei
  \[
    \ad_x \coloneqq [x,-] \colon V \to V, \quad y \mapsto [x,y].
  \]
  Zeigen Sie, dass die folgenden beiden Aussagen äquivalent sind:
  \begin{enumerate}
    \item
      Die alternierende Bilinearform $[-,-]$ erfüllt die Jacobi-Identität, d.h.\ es ist
      \[
        [x,[y,z]] + [y,[z,x]] + [z,[x,y]] = 0
        \quad
        \text{für alle $x, y, z \in V$}.
      \]
    \item
      Es gilt $\ad_x([y,z]) = [\ad_x(y), z] + [y, \ad_x(z)]$ für alle $x, y, z \in V$.
      (Man sagt, dass $\ad_x$ eine Derivation bezüglich $[-,-]$ ist.)
  \end{enumerate}
\end{question}






\newpage


\printsolutions



\end{document}
