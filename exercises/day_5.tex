\documentclass[a4paper, 10pt]{scrartcl}

\usepackage{../generalstyle}
\usepackage{exercisestyle}


\title{Lineare Algebra II Repetitorium \\ Übungen, Tag 4}
\author{Jendrik Stelzner}
\date{\today}


\begin{document}
\maketitle


\begin{question}
  Entscheiden Sie, welche der folgenden Aussagen für alle $n \geq 1$ gelten.
  \begin{enumerate}[leftmargin=*]
    \item
      Die Wegzusammenhangskomponenten von $\GL_n(\Reals)$ sind die beiden Untergruppen
      \begin{gather*}
        \GL_n(\Reals)_+ = \{ S \in \GL_n(\Reals) \mid \det S > 0 \}
      \shortintertext{und}
        \GL_n(\Reals)_- = \{ S \in \GL_n(\Reals) \mid \det S < 0 \}.
      \end{gather*}
    \item
      Für alle $A, B \in \GL_n(\Reals)$ liegen entweder $A$ und $B$ in derselben Zusammenhangskomponente, oder $A$ und $-B$ liegen in derselben Zusammenhangskomponente.
    \item
      Von den beiden Wegzusammenhangskomponente von $\GL_n(\Reals)$ ist $\Orthogonal(n)$ diejenige, welche die Einheitsmatrix enthält.
    \item
      Die schiefsymmetrischen Matrizen $\mathfrak{o}_n(\Reals) = \{ A \in \Mat_n(\Reals) \mid A^T = -A \}$ bilden eine wegzusammenhängende und abgeschlossene Teilmenge von $\Mat_n(\Reals)$.
    \item
      Ist $n \geq 2$, so hat die Gruppe $\Unitary(n) \cap \GL_n(\Reals)_+$ genau zwei Wegzusammenhangskomponenten.
    \item
      Es ist $G = \{ S \in \GL_n(\Reals) \mid S^{-1} = -S \}$ eine zusammenhängende, aber nicht wegzusammenhängende Untergruppe von $\GL_n(\Reals)$.
    \item
      Jede Untergruppe von $\GL_n(\Complex)$ ist wegzusammenhängend.
    \item
      Die Menge der Drehmatrizen
      \[
        D
        \coloneqq
        \left\{
          \begin{pmatrix*}[r]
            \cos \varphi  & -\sin \varphi \\
            \sin \varphi  &  \cos \varphi 
          \end{pmatrix*}
        \,\middle|\,
        \varphi \in \Reals
        \right\}
      \]
      ist eine wegzusammenhängende Untergruppe von $\GL_2(\Reals)$.
  \end{enumerate}
\end{question}


\begin{question}
  Es sei $n \geq 1$.
  \begin{enumerate}[leftmargin=*]
    \item
      Zeigen Sie, dass $\Reals \smallsetminus \{0\}$ nicht zusammenhängend ist.
    \item
      Folgern Sie, dass $\GL_n(\Reals)$ und $\Orthogonal(n)$ nicht zusammenhängend sind.
    \item
      Wieso lassen sich die obigen Argumente nicht zu $\Korper = \Complex$ verallgemeinern?
  \end{enumerate}
\end{question}


\begin{question}
  Es sei $V$ ein dreidimensionaler euklidischer Vektorraum und $d \colon V^{\times 3} \to \Reals$ eine alternierende Trilinearform.
  \begin{enumerate}[leftmargin=*]
    \item
      Zeigen Sie, dass es für alle $v_1, v_2 \in V$ genau ein $v_1 \times v_2 \in V$ gibt, so dass
      \[
        \bracket{v_1 \times v_2, w} = d(v_1, v_2, w)
        \quad
        \text{für alle $w \in V$}.
      \]
    \item
      Zeigen Sie, dass $- \times - \colon V \times V \to V$ bilinear und alternierend ist.
  \end{enumerate}
  Es sei nun $(v_1, v_2, v_3)$ eine Orthonormalbasis von $V$, so dass $d(v_1, v_2, v_3) = 1$.
  \begin{enumerate}[leftmargin=*, resume]
    \item
      Zeigen Sie, dass $v_1 \times v_2 = v_3$, $v_1 \times v_3 = -v_2$ und $v_2 \times v_3 = v_1$.
    \item
      Folgern Sie, dass allgemeiner
      \begin{align*}
         &\,  (a_1 v_1 + a_2 v_2 + a_3 v_3) \times (b_1 v_1 + b_2 v_2 + b_3 v_3)          \\
        =&\,  (a_2 b_3 - a_3 b_2) v_1 + (a_3 b_1 - a_1 b_3) v_2 + (a_1 b_2 - a_2 b_1) v_3.
      \end{align*}
  \end{enumerate}
\end{question}



\begin{question}
  Es sei $V$ ein $K$-Vektorraum und $[-,-] \colon V \times V \to V$ eine alternierend bilineare Abbildung.
  Für jedes $x \in V$ sei
  \[
    \ad_x \coloneqq [x,-] \colon V \to V, \quad y \mapsto [x,y].
  \]
  Zeigen Sie, dass die folgenden beiden Aussagen äquivalent sind:
  \begin{enumerate}[leftmargin=*]
    \item
      Die alternierende Bilinearform $[-,-]$ erfüllt die Jacobi-Identität, d.h.\ es ist
      \[
        [x,[y,z]] + [y,[z,x]] + [z,[x,y]] = 0
        \quad
        \text{für alle $x, y, z \in V$}.
      \]
    \item
      Es gilt $\ad_x([y,z]) = [\ad_x(y), z] + [y, \ad_x(z)]$ für alle $x, y, z \in V$.
      (Man sagt, dass $\ad_x$ eine Derivation bezüglich $[-,-]$ ist.)
  \end{enumerate}
\end{question}


\begin{question}
  Es sei $V$ ein euklidischer Vektorraum und $A, B \in V$ seien zwei lineare unabhängige Vektoren.
  Zeigen Sie, dass es genau einen normierten Vektor $\tangent_{AB} \in V$ mit den folgenden Bedingungen gibt:
  \begin{itemize}
    \item
      Es ist $\tangent_{AB} \in \Ell(A, B)$.
    \item
      Es gilt $\tangent_{AB} \perp A$.
    \item
      Es gilt $\bracket{\tangent_{AB}, B} > 0$.
  \end{itemize}
  Skizzieren Sie die Situation.
\end{question}










\newpage


\printsolutions



\end{document}
